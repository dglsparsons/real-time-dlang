In order to supply support for programming real-time systems, we must first 
consider the characteristics of a real-time system. This section aims to outline
any necessary primitives for a programming language, explaining why they are 
necessary. 

\subsection{Threads and Interprocess Communication}
Real-time systems are inherently parallel. They are concerned heavily with 
the notion of timing and scheduling many different components across a system. 
The ability to schedule these components in isolation through the medium 
of threads therefore provides a valuable tool in reducing the complexity in 
constructing the system, and additionally increases the readability of the 
final system 
\cite{BURNS +WELLING p132}. It is therefore a requirement of a real-time systems 
programming language to provide support for an abstraction into isolating 
concurrent events and their control into separate entities. Through Java, and 
C with POSIX threads, this isolation is provided through threads or processes, 
and in Ada is provided with Tasks 
\cite{books.google.co.uk/books?id=ZDRBv8U_bJ0C&pg=PA251}.

\subsection{Priority Scheduling}
In addition to the creation and isolation of processes through the use of 
threads, on a similar abstraction, it is a requirement of real-time systems 
that control over their scheduling is provided. 
While many algorithms exist that enable real-time systems to be scheduled, such 
as earliest deadline first (EDF), priority based scheduling approaches are the 
most commonly used approach 
\cite{http://research.cs.queensu.ca/home/akl/techreports/scheduling.pdf}. 
It is therefore a requirement of the programming language to provide support 
for a priority based scheduling algorithm, or allowing the scheduler to be 
selected. Additional complication is introduced when we consider if scheduling 
is the domain of the programming language, or of the operating system:
threads may be either language controlled, or operating system controlled. 
If threads are language controlled, then it is necessary to provide 
language functionality enabling threads to have priorities, and scheduling threads 
based on their priority.
If the threads are operating system controlled, then the language needs 
support for interacting with the underlying operating system in order to 
adjust the scheduler. 

\subsection{Bounded Priority Inversion}
When using shared resources in a priority based system it is possible for 
a high priority task to become blocked from making progress due to a low
priority task holding a resource. Additionally, the low priority task may be 
preempted, and prevented from making progress. In this instance, the high priority task is 
effectively running at the lowest priority of the system: it has to wait for all other 
tasks to complete before it can progress. This is a well studied phenomenon known 
as priority inversion 
\cite{Burns and Welling page..}. 
In order to provide a reliable response in a priority based system, it is 
therefore necessary that priority inversion is bounded.
Two protocols exist that solve this problem: the priority inheritance protocol 
and the priority ceiling protocol. Each protocol has its own respective 
strengths and weaknesses, but enable reliable behaviour of the system 
\cite{books.google.co.uk/books?id=coPT7vaEjFsC&pg=PA87}.
It is therefore a requirement of a real-time systems programming language 
that the underlying data sharing system is not subject to unbounded priority 
inversion: the ability to specify the use of either the priority inheritance 
protocol or the priority ceiling protocol is a necessary feature.  

\subsection{Monotonic Clocks and Absolute Sleep}
%TODO - fix this section
In general programming, when a worker thread has no computation to be 
performing, a call is made to a sleep method. This will cause the thread
to hand over any computation resources for a defined period of time. In a 
real-time system however, we often require fine-grained control of exactly
when a thread will awaken. The typical approach involving a relative sleep 
time can be imprecise due to preemption. In order to achieve a predictable 
sleep function, as required in a real time system, it is therefore necessary 
to have the ability to wait until an absolute time. Further complication 
occurs with the definition of time: many implementations of time are subject 
to phenomena such as daylights savings time, or leap seconds. 
In order to counteract these, a Monotonic clock must be used. This ticks at a 
predefined constant rate, giving predictable behaviour. 

\subsection{Asynchronous Transfer of Control}
Asynchronous Transfer of Control (ATC) is a controversial topic with regards to its 
implementation in many languages, yet it is deemed a useful, if not necessary 
feature of a real-time system 
\cite{books.google.co.uk/books?id=pD9sCQAAQBAJ&pg=PA113}. ATC is the transfer 
of control within a thread, that has been triggered by an external event or thread, 
its effect may be instantaneous, or deferred until a safe section has been reached. 
In addition, sections of code susceptible to ATC may be nested, causing a 
complex problem in terms of semantics, methodology, and implementation. 
However, the use of ATC extends to many purposes such as low latency responses to events, timing 
out a thread of computation, or terminating threads, making it an invaluable feature. 
