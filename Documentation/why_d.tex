This sections aims to outline the reasons for using D as a programming language,
specifically in regard to Real Time systems. 

\subsection{High and Low Level}
% Aims to achieve everything that is possible with other languages such as C++
% Contains high level abstractions
The D programming language is primarily inspired by C++, aiming to be a "C++ done
right" \cite{qznc.github.io/d-tut/philosophy.html}. This explicates the 
power that the language possesses: D is primarily a high level language: 
abstractions to aid in speed and ease of development are rife, and can 
be seen in the expansive standard libraries that are provided. However, D still 
maintains low level capabilities: the language still has access to inline assembler, 
system calls, interrupts, and low level I/O as well as being a compiled language. 
\par\bigskip\noindent
The flexibility of D is one of the largest advantages. It "allows its user to tap 
into a wide range of programming techniques, without throwing too many 
idiosyncrasies in the way" \cite{ddili.org/ders/d.en/Programming_in_D.pdf}, 
while simultaneously generating code that can be aggressively optimized by the
compiler while still being idiomatic\cite{dlang.org/overview.html}.

\subsection{Memory Safe Programming}
% Concept of memory safe programming that D supports.
An area of particular interest to Real Time systems is the notion of memory 
safe programming. In addition to providing powerful in built types for arrays 
in D, the D compiler implements range checking on compilation. 
Additional safety features are also provided in the safeD subset of the language
\cite{dlang.org/safed.html}. This will only allow functions to perform 
operations that are deemed as either safe or trusted, preventing any 
potential corruption of memory.

\subsection{Unittesting}
% Idea of built in tests, contract programming, debugging, error handling
Another unique feature of D is the ability to build unit tests into a 
program without the need for external libraries: D provides support for 
unittesting as a language feature. 

\subsection{Additional Interesting Features}

D is of further interest for two rather unusual features of the language.
First of all, an unusual approach is taken to sharing memory between threads: 
the concept of Thread Local Storage (TLS) is used. 
Rather than each thread having access to shared resources, resources are by 
default duplicated in each new thread created. In order to share resources they 
have to be explicitly specified as \texttt{shared}. This approach is taken in 
order to both reduce the use of global variables, and reduce any errors that may
occur as a result of this. \cite{dlang website for migrating to shared}
Another interesting feature is the atomic section of the core library, provided 
by the druntime. This documents a set of uninterruptable actions that can be taken, 
enabling a lock free approach to managing shared resources. \cite{shared documentation}.

