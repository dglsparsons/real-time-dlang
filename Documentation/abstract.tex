% Abstract to go at the start of my report. 
% Word Limit: 500 words. 
% May also have some ethics statement, etc... 
% Whatever that is supposed to mean...

Real-time systems are commonplace and are found in a wide range of 
situations such as manufacturing processes, automotive vehicles and embedded 
sensors. As a result, they are an area of increasing importance in the field 
of Computer Science. Generally stated, a real-time system is a type of information 
processing system in which a response to an external input stimuli 
has to be generated within a fixed time period. Correctness of the 
system may depend not only on the result, but also the time that it was delivered. 
\par\bigskip\noindent
The development of real-time systems typically involves the use of a programming 
language providing explicit support for real-time systems. 
This support involves the provision of protocols that allow the system to 
behave in a more reliable manner, such as Asynchronous Transfer of Control
(ATC), or bounded priority inversion. 
My project involves the provision of such support to a new programming language: 
D. D currently has very limited support for the development of real-time systems. 
\par\bigskip\noindent
Several programming languages already exist that provide this support:
the most notable of which are 
Real-time Java, Ada and C with POSIX Threads. These languages are well researched, 
with a rich field of study comparing their approaches and behaviour. 
However, comparatively little research has gone into providing new programming 
languages with support for the development of real-time systems. My project 
targets this omission by extending real-time support to a previously unsupported 
language.
\par\bigskip\noindent
By examining existing programming languages and research on real-time systems, a core set 
of functionality for any real-time system supporting language was defined. 
In some aspects, such as concurrent programming, support was already provided in D. 
However, for the majority of the features under consideration, additional support was required. 
Different approaches for their implementation were considered, and functionality 
was implemented as a library. 
With regards to ATC, there was no clear 
best approach. As a result, two alternative approaches have 
been considered, implemented, and profiled. A comparison between the two approaches 
has shown that an exception based approach leads to a higher performance 
solution. 
\par\bigskip\noindent
Furthermore, an experiment compares the behaviour of a D program written 
using the implemented ATC approaches with an equivalent 
real-time Ada program. This found that the performance of ATC in the two 
programming languages is roughly comparable. 
\par\bigskip\noindent
As a result of this project, the pool of applicable languages for the development 
of real-time systems has been extended to incorporate a new and powerful language, D, which 
boasts high performance and strong safety. 
