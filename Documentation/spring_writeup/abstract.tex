% Abstract to go at the start of my report. 
% Word Limit: 500 words. 
% May also have some ethics statement, etc... 
% Whatever that is supposed to mean...

Real-time systems are an area of increasing relevance and interest in the field 
of Computer Science. Generally stated, a real-time system is a type of information 
processing system in which a response to an external input stimuli 
has to be generated within a fixed time period. Correctness of the 
system may depend not only the result, but also the time that it was delivered. 
Such systems are commonplace and are found in a wide range of 
situations such as manufacturing processes, automotive vehicles and embedded 
sensors.
\par\bigskip\noindent
The development of real-time systems typically involves the use of a programming 
language providing explicit support for real-time systems. 
This support involves the provision of protocols that allow the system to 
behave in a more reliable manner, such as asynchronous transfer of control or 
bounded priority inversion. 
My project entails the provision of such support to a new programming language: 
D. D currently has very limited support for the development of real-time systems. 
\par\bigskip\noindent
Several programming languages already exist that provide support for the 
development of real-time systems. The most notable of which are 
Real-time Java, Ada and C with POSIX Threads. These languages are well researched, 
with a rich field of study comparing their approaches and behaviour. 
However, comparatively little research has gone into providing new programming 
languages with support for the development of real-time systems. My project 
targets this omission by extending real-time support to a previously unsupported 
language.
\par\bigskip\noindent
By examining existing programming languages and research on real-time systems, a core set 
of functionality for any real-time system supporting language was defined. 
In some aspects, such as concurrent programming, support was already provided in D. 
However, for the majority of real-time programming language features,
additional support was required. Different approaches for the implementation 
of these features was considered, and functionality was implemented into 
a library. With regards to asynchronous transfer of control, there was no clear 
approach for implementation. As a result, two alternative approaches have 
been considered, profiled and implemented. A comparison between the two approaches 
has shown that a POSIX signal based approach leads to a higher performance 
solution. 
\par\bigskip\noindent
Furthermore, a series of experiments compares the behaviour of a D program written 
using the added features with an equivalent real-time Ada program. This found 
that\ldots. 
\par\bigskip\noindent
As a result of this project, the pool of applicable languages for the development 
of real-time systems has been extended to incorporate a new and powerful language, D, which 
boasts high performance and strong safety. 


%However, there is no reason that this pool of applicable languages can not be extended to 
%incorporate new, powerful programming languages. 
%D is one such language, aiming for a C++ style syntax, coupled with powerful 
%abstractions to ease development and the ability to take advantage of modern 
%compiler technology to provide aggressively optimised code.
%The D programming language is of particular interest for 
%the development of real-time applications due to it's combined ease of development
%and rich feature set. 

%However, it is noteworthy that D does not provide support for real-time systems: 
%it has the required characteristics of enabling concurrent programming and low-level 
%control, but lacks required features. 
%In order to develop a real-time application or system, additional functionality 
%must therefore be provided.
%This project intends to contribute to the pool of languages supporting the development 
%of real-time systems and applications by providing necessary features 
%to the D Programming language. 


