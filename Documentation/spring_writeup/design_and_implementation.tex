%
%
%
This chapter aims to describe the implementation of the project requirements into the D
programming language, highlighting the completed implementations and any  
decisions made during their design.

\section{Method of Provision}
For the end goal of this project to be usable, various methods for its delivery 
were considered. There are two significant alternatives. 
If the compiler or the 
core runtime features of the language do not need to be altered to provide the 
desired functionality, the project may be packaged as an external library. 
This is the most desirable end goal, as it succeeds in meeting requirement 8. 
Failing this, a modified runtime and/or compiler would need be provided.
However, this system would require maintenance to keep it up to date with the
core set of D functionality. 

\section{Requirement 1: Concurrency}
D, being a C/C++ style language, provides support for the creation and use of 
threads. Threads in D follow a Java style approach: \texttt{Thread} is a class, 
allowing subclasses to be derived. The default 
class method will allow a function to be passed by reference and executed. 
These features can be seen in the example below, taken from the D website
\cite{core-thread}: 
%TCusedtoignorethis
\begin{lstlisting}[basicstyle=\small]
import core.Thread; 

class DerivedThread : Thread
{
    this()
    {
        super(&run);
    }
    private:
    void run()
    {
        // Derived thread running.
    }
}

void threadFunc()
{
    // Composed thread running.
}

// create and start instances of each type
void main()
{
    auto derived = new DerivedThread();
    auto composed = new Thread(&threadFunc);
    derived.start; 
    composed.start;
}
\end{lstlisting}
%TCusedtoendignorethis
It is therefore evident that D provides 
concurrent control of system components. Therefore, for requirement 1, no
further implementation is required. 

\section{Requirement 2: Priority Scheduling}
The ability to correctly schedule tasks or threads based on their priority is of 
critical importance to a real-time system. D's \texttt{Thread} class contains the
parameter, \texttt{priority}, which can be used to adjust or retrieve 
the \texttt{Thread's} priority \cite{core-thread}. 
However, in order for priority based scheduling to be used, this alone is not 
enough: the system scheduler must also be changed.
No support exists within D's standard libraries to set the scheduler, and therefore, 
this support must be supplied. 
This is achieved through interaction with standard C libraries. 
The Linux function \texttt{sched\textunderscore{}setscheduler} allows the operating system 
scheduler to be changed to either a first-in-first-out or round robin scheduler \cite{sched-setscheduler}. 
In order to provide an intuitive interaction, a wrapper has been written around
this system call, providing a more 'D like' interface and a level of
abstraction. This takes the following form:
\begin{lstlisting}[basicstyle=\small,language=C++]
public import core.sys.posix.sched 
                : SCHED_FIFO, SCHED_OTHER, SCHED_RR; 

void setScheduler(int scheduler_type, int scheduler_priority)
{
    import core.sys.posix.sched 
                : sched_param, sched_setscheduler; 

    sched_param sp = { 
        sched_priority: scheduler_priority 
    }; 

    int ret = sched_setscheduler(0, scheduler_type, &sp); 
    if (ret == -1) {
        throw new Exception("scheduler did not properly set");
    }
}

void setFIFOScheduler(int schedPriority)
{
    setScheduler(SCHED_FIFO, schedPriority);
}

void setRRScheduler(int schedPriority)
{
    setScheduler(SCHED_RR, schedPriority); 
}
\end{lstlisting}
This functionality may then be used in the following manner: 
\begin{lstlisting}[basicstyle=\small]
void main()
{
    setFIFOScheduler(50); 
}
\end{lstlisting}
The ability to set the scheduler to use a priority based method, combined with 
D's ability to set the priority of a \texttt{Thread}, enables a 
priority based scheduling approach to be used. 
For some operating systems, setting the scheduler may require super user privileges. 
Furthermore, this project is interested only in operating systems capable of meeting tight timing 
requirements in any calls to the kernel and context switching.
If using Linux, the underlying operating system may require patching in order 
to be fully viable for a real-time system. Because of this, real-time patches 
to the Linux kernel exist \cite{rt-wiki}. 
Furthermore, special care must also be taken when developing such a system;
selecting appropriate hardware and programming the system in a manner appreciative 
of the underlying hardware \cite{rt-wiki-how-to}.


\section{Requirements 3 and 4: Monotonic Clocks and Absolute Sleep}
As mentioned in the previous section, it is a necessary to be able to sleep until 
an absolute time. This sleep must also use a monotonic clock 
to remove any subjection to clock-drift or time-zone changes. D has support 
for a monotonic clock as part of its core libraries 
\cite{dlang-core-time}. This is accessible using 
\texttt{MonoTime}. However, there is no ability to sleep until an 
absolute time. Using D's ability to interact with C libraries, it is possible 
to leverage C's ability to perform this operation. The C function 
\texttt{clock\textunderscore{}nanosleep} allows an absolute time to be specified
\cite{clock-nanosleep}.
In order to provide a more usable interaction, this project provides a wrapper
function. 
This function accepts the D type \texttt{MonoTime} as an input, converts it to the C 
equivalent, and calls the C function \texttt{clock\textunderscore{}nanosleep}, 
allowing an absolute delay. The code is displayed below: 
\begin{lstlisting}[basicstyle=\small]
void delayUntil(MonoTime timeIn)
{
    import core.sys.linux.time; 
    import core.time : Duration, timespec; 
    Duration dur = timeIn - MonoTime(0) ;
    long secs, nansecs; 
    dur.split!("seconds", "nsecs")(secs, nansecs); 
    timespec sleep_time = timespec(secs, nansecs); 
    if (clock_nanosleep(CLOCK_MONOTONIC, TIMER_ABSTIME, 
                                    &sleep_time, null)) {
        throw new Exception("Failed to sleep as expected!"); 
    }
}
\end{lstlisting}
This allows simplistic use for a real-time application. An absolute sleep can 
then be performed in the following manner: 
\begin{lstlisting}[basicstyle=\small]
void main()
{
    auto time = MonoTime.currTime; 
    time += 3.seconds; 
    delayUntil(time);
}
\end{lstlisting}
As such, this project provides support for pausing a \texttt{Thread's}
execution until an absolute time. 

\section{Requirements 5 and 6: Bounded Priority Inversion}
In D, the \texttt{Mutex} class follows a similar approach to the \texttt{Thread}
class: the language's implementation provides a wrapper around operating system 
calls. However, this \texttt{Mutex} class has no 
inherent ability to provide the Priority Inheritance or the Priority Ceiling protocol. 
Real-time systems have been shown to only be viable on operating systems that 
are POSIX compliant, and due to the quantity of work involved in reimplementing 
and defining these protocols, the decision was made to expand the languages
inbuilt POSIX mutexes. The priority inheritance and priority ceiling protocols 
are available when using POSIX mutexes in C \cite{mutex-setprotocol}. 
\par\bigskip\noindent
As requirement 8 highlights the importance of an unmodified runtime, the 
constructor for \texttt{Mutex} class cannot be extended. Instead it must be fully 
reimplemented as a new class with the constructor adding 
the C function \texttt{pthread\textunderscore{}mutexattr\textunderscore{}setprotocol} 
\cite{mutex-setprotocol}. 
The full implementation of this \texttt{RTMutex} class is defined:
\lstinputlisting[basicstyle=\small,language=Java]{rtmutex.d}
A mutex using the inheritance protocol may then be created and initialised in 
the following manner: 
\begin{lstlisting}[basicstyle=\small]
auto a = new RTMutex(PRIORITY_INHERIT);
\end{lstlisting}
However, this is not a very D-like implementation: \texttt{enums} are not widely used 
in constructors. Instead, as with the scheduler calls, 
it is more intuitive to wrap this into two separate classes. Additionally, 
further support 
is needed to set and retrieve the priority ceiling associated with the 
mutex. 
This is achieved by a second class that provides access to an \texttt{RTMutex}
instance alongside a wrapper around the C functions 
\texttt{pthread\textunderscore{}mutex\textunderscore{}getprioceiling} and 
\texttt{pthread\textunderscore{}mutex\textunderscore{}setprioceiling}.
This is implemented in the following manner: 
\begin{lstlisting}[basicstyle=\small]
class CeilingMutex 
{
    private import core.sync.\texttt{Exception} : SyncError;
    alias ceilingMutex this;
    RTMutex ceilingMutex;

    // Initializes a new CeilingMutex
    this()
    {
        ceilingMutex = new RTMutex(PROTOCOL_CEILING);
        this.ceiling = 1;
    }

    final @property int ceiling()
    {
        int ceiling; 
        if(pthread_mutex_getprioceiling(this.handleAddr, 
                                                &ceiling))
            throw new SyncError("Unable to fetch the priority 
                           ceiling."); 
        return ceiling; 
    }

    final @property void ceiling(int val)
    {
        if(pthread_mutex_setprioceiling(this.handleAddr, 
                                                val, null))
            throw new SyncError("Unable to set the priority 
                           ceiling."); 
    }
}
\end{lstlisting}
This allows the priority ceiling to be set or retrieved as if it were a property of 
the \texttt{RTMutex} class. Similarly, the functions defined in \texttt{RTMutex} 
can be accessed as though they were part of the \texttt{CeilingMutex} class. 
A similar result is achieved for the \texttt{InheritanceMutex} class.
This allows an idiomatic and readable interaction: 
\begin{lstlisting}[basicstyle=\small]
auto a = new CeilingMutex;
a.ceiling = 50; 
synchronized(a) 
{
    // perform an exclusive action
}
\end{lstlisting}
Therefore, on a POSIX compliant operating system, D's \texttt{Mutex} class can 
be reimplemented, incorporating both the priority inheritance and 
priority ceiling protocols. 

\section{Requirement 7: Asynchronous Transfer of Control} % 600 words
The final primitive required for a real-time system is the ability to perform an Asynchronous Transfer of 
Control (ATC). This entails providing a section of code that can be interrupted 
and aborted asynchronously.
As detailed in the previous section, there are several different methods of 
achieving this aim. Here, each method is considered and the logic behind the 
implementation explained. 

\subsubsection*{Setjmp and Longjmp} 
In C, ATC can be achieved through use of the functions, \texttt{setjmp} and 
\texttt{longjmp}. These allow 
the current execution of a program to be altered by `jumping' to a saved
point in the execution.
This jump may occur asynchronously through the use of POSIX signals 
and a signal handler. 
However, when using this approach, even though the functions are thread and
signal safe \cite{setjmp}, it is possible for the stack to become corrupted in
D, as with C++ \cite{unwinding-stack}. 
During ATC, the stack may be altered as functions are called, and returning to a 
previous location without correctly rewinding the stack may cause memory corruption. 
As a result of this poor memory management, this method was 
not implemented in the end product.

\subsubsection*{Thread Cancellation}
The second approach considered for the implementation of ATC in D is a thread 
cancellation approach. When using POSIX threads, it is possible to terminate a 
thread during its execution using the \texttt{pthread\textunderscore{}cancel} function. 
In order to neatly abstract this cancellation, an \texttt{Interruptible} class
was defined.
Initialising this will create a new \texttt{Thread}, 
inheriting the priority of the calling \texttt{Thread}. 
On calling a \texttt{start} function, an abortable section of code will begin 
execution in this \texttt{Thread}, and the calling
\texttt{Thread} will become blocked. 
For the asynchronous interrupt mechanism, using
\texttt{pthread\textunderscore{}cancel},
two alternative approaches were considered:
the first follows the Java implementation of a 
\texttt{Thread.interrupt} method; 
the second approach instead places this \texttt{interrupt} method as part of the 
\texttt{Interruptible} class. 
While the difference between the two approaches may appear small, it is significant: 
in the first implementation, it is not clear which \texttt{Interruptible}
section is to be cancelled in a situation where the \texttt{Interruptible} 
sections are nested. 
Due to this ambiguity, a decision was made to instead implement the second
approach. 
For example, this can be used to provide a 2 second timeout on a function call:
\begin{lstlisting}[basicstyle=\small]
void interruptibleSection()
{
    while(true)
    {
        // loop forever.
    }
}

void main()
{
    auto a = new Interruptible(&interruptibleSection); 
    new Thread({
        Thread.sleep(2.seconds);
        a.interrupt;
    }).start;
    a.start; 
}
\end{lstlisting}
\par\bigskip\noindent
So far, it has been assumed that the interrupt function performs the exact 
functionality required. 
However, using \texttt{pthread\textunderscore{}cancel} alone does not guarantee 
an immediate cancellation: it is only guaranteed 
to cancel once both a signal is received and a cancellation point in the code 
is reached. 
Many standard C functions are defined as being thread cancellation points, but 
there is no guarantee that these may be called within the target thread 
\cite{pthread-cancel-points}. It is possible to set the 
cancellation of a thread to be immediate through the function 
\texttt{pthread\textunderscore{}setcanceltype}, by setting the value to 
\texttt{PTHREAD\textunderscore{}CANCEL\textunderscore{}ASYNCHRONOUS}. This is 
crucial in either tight loops or where cancellation must be immediate. 
\par\bigskip\noindent
However, this immediate cancellation comes at the cost of safety: interrupting a 
thread during the middle of a function call, such as a memory allocation, 
may leave memory in an inconsistent state or crash the program. 
This gives rise to the requirement of having regions of code in which
interrupts are deferred. 
Two alternative approaches were considered for the implementation 
of this feature: a method of the 
\texttt{Interruptible} class that toggles the cancellation of the thread  
between \texttt{PTHREAD\textunderscore{}CANCEL\textunderscore{}ENABLE}
and \texttt{PTHREAD\textunderscore{}CANCEL\textunderscore{}DISABLE}; or,
to toggle a boolean flags in the \texttt{Interruptible} class indicating
whether it is possible to interrupt. 
Using either approach, any interrupt that arrives while interrupts are deferred
must be stored. When interrupts are re-enabled, this may then trigger and
cancel the \texttt{Thread}. 
%TODO - current progress
Conceptually, the two approaches are similar. However, the first approach does 
not provide a viable method for nesting \texttt{Interruptible} classes. As it is 
possible for a second interruptible class to be nested within the first, setting 
\texttt{PTHREAD\textunderscore{}CANCEL\textunderscore{}DISABLE} may cause an 
inner \texttt{Interruptible} to be non-interruptible, even though its outer parent 
has been cancelled. In order to provide a fine-grained control over the two, 
even in the cases where nesting is possible, both these methods are implemented. 
Basic deferral can be achieved through the \texttt{deferred} properties. In order 
to execute code safely, even in nested cases, an \texttt{executeSafely} method that 
defers the cancellation has been implemented as follows. 
\begin{lstlisting}[basicstyle=\small]
class Interruptible
{
    ..
    void executeSafely(void delegate() fn)
    {
        if (pthread_setcancelstate(PTHREAD_CANCEL_DISABLE, 
                                                    null))
        {
            throw new Error("Unable to set thread cancellation 
                                                    state");
        }
        fn();
        if (pthread_setcancelstate(PTHREAD_CANCEL_ENABLE, 
                                                      null))
        {
            throw new Error("Unable to set thread cancellation 
                                                    state");
        }
    }
}
\end{lstlisting}
this may then be used in the following manner. 
\begin{lstlisting}[basicstyle=\small]
alias getInt = Interruptible.getThis;
void interruptibleFunction() 
{
    for(int i = 0; i < 10; i++)
    {
        void update() 
        {
            // perform some memory allocation
            void* x = GC.malloc(10_000); 
        }
        getInt.executeSafely(&update);
    }
}
\end{lstlisting}
%TCusedtoendignorethis
This provides a safe mechanism for handling memory allocation or critical sections 
of code. Furthermore, this enables code to be executed safely despite being 
nested in outer \texttt{Interruptible} classes. 
\par\bigskip\noindent
Additionally, it is often desirable to execute code once the cancellation has taken 
place. In C with POSIX threads, this can be achieved through the use of the 
\texttt{pthread\textunderscore{}cleanup\textunderscore{}push} function. 
This is achievable through use of D's interoperability with C. Again, functions are provided 
that wrap this functionality, providing a more usable interface. An example of using 
this cleanup code is as follows: 
%TCusedtoignorethis
\begin{lstlisting}[basicstyle=\small]
__gshared Interruptible a; 

extern (C) void thread_cleanup(void* arg) nothrow
{
    writeln("Cleanup Function");
}

void interruptibleFn()
{
    auto a = addCleanup(&thread_cleanup, cast(void*)void);
    while(true)
    {
        // loop forever
    }
}
\end{lstlisting}
%TCusedtoendignorethis
This provides the desired functionality of ATC, as set out in requirement 7, 
through the use of thread cancellation in a two-thread model.

\subsubsection*{Exceptions and Signals}
An alternative approach to the implementation of ATC is to use D's inbuilt \texttt{Exception}/error 
handling mechanisms. Errors and \texttt{Exception}s are typically used explicitly within code 
to affect the flow of control within a program.
For a POSIX system, it is possible to asynchronously invoke code remotely using 
signals and signal handlers. While typically used on a process wide level, it
is also possible to invoke these signal handlers for a specific thread through
the use of a C function, \texttt{pthread\textunderscore{}kill}. By using a
real-time signal, such as SIGRTMIN, there are guarantees that the signal
arrives in a timely manner and that any signals are sent in the correct order. 
Thus, in D, it is possible to remotely invoke code in a \texttt{Thread} through 
the use of signals and signal handlers. This can be also be used to remotely
throw an \texttt{Exception}. The following example shows a simple invocation of ATC
using achieved by throwing and catching an \texttt{Exception}, breaking out of an
infinite loop: 
%TCusedtoignorethis
\begin{lstlisting}[basicstyle=\small]
import core.sys.posix.pthread, 
       core.sys.posix.signal, 
       std.stdio;

Exception ex = new Exception("Remotely Triggered Exception"); 

extern (C) void sig_handler(int signum) @nogc nothrow
{
    throw ex;
}

void setupSignalHandler()
{
    sigaction_t action; 
    action.sa_handler = &sig_handler; 
    sigemptyset(&action.sa_mask);
    sigaction(36, &action, null); 
}

void threadFunction()
{
    writeln("This is the thread"); 
    while(true)
    {
        Thread.sleep(1.seconds);
    }
}

void main()
{
    setupSignalHandler; 

    auto a = new Thread(&threadFunction); 
    a.start; 

    Thread.sleep(1.seconds); 
    pthread_kill(a.id, 36); 
}
\end{lstlisting}
%TCusedtoendignorethis
The above example displays the simple principal behind this method of ATC.
However, there are many aspects of its behaviour that make the above
implementation impractical. 
First, ATC sections may be nested inside each other. Using the above 
method, it would not be possible to exit two ATC sections through a single interrupt
to the outermost section. This occurs because the \texttt{Exception} would be caught and
handled in the inner ATC section. 
Secondly, internal components may trigger \texttt{Exception}s or may have \texttt{Exception} handlers 
of their own. Using this approach, an \texttt{Exception} that is intended to perform ATC 
may instead be caught and would thus have no effect. 
In order to prevent this from happening, the final implementation for this
project does not use \texttt{Exceptions}, but instead uses D's notion of an
\texttt{Error}. Conceptually, \texttt{Errors} and \texttt{Exceptions} are
similar, both being throw and catch-able objects. 
However, they are used differently within D: \texttt{Exceptions} are used to
manage the control of flow under \texttt{Exception}al circumstances, whereas
\texttt{Errors} are typically used for terminating the program and tracing faults. 
However, the two also differ fundamentally. 
\texttt{Errors} are allowed to propagate through generic \texttt{Exception} catches, and thus propagate 
through all \texttt{Exception} handlers. In this regard, they exhibit the desired
behaviour, enabling abortable code to use \texttt{Exceptions} freely. The only
limitation here is that the end user does not program the
abortable code to catch all \texttt{Errors}. As errors generally occur when the
program is unable to run or corrupts memory, it is not advisable to do this. 
For Signal and \texttt{Exception} based ATC, rather than using a generic
\texttt{Error}, an \texttt{ATCInterrupt} class is defined. 
This class extends the \texttt{Error} class.
\par\bigskip\noindent
In order to allow nesting of Interruptible sections of code, the \texttt{Error} thrown
must also be rethrown if it is not caught by its corresponding catch statement. 
This is achieved by creating a new error for each interruptible section, and
this error having a property that is checked in the catch statement. The error
is then rethrown if it is not caught by its owner. 
This is achieved as follows: 
%TCusedtoignorethis
\begin{lstlisting}[basicstyle=\small]
private class ATCInterrupt : Error
{
    Interruptible owner;
    this(Interruptible own)
    {
        super(null, null);
        owner = own;
    }
}

class Interruptible
{
    private ATCInterrupt error; 
    private void function() fn; 

    this(void function() func)
    {
        error = new ATCInterrupt(this); 
        fn = func
    }

    private ATCInterrupt caughtInt;

    void start()
    {
        try
        {
            fn();
        }
        catch (ATCInterrupt ex)
        {
            if (ex.owner != this)
                caughtInt = ex;
        }
        finally
        {
            if (!(caughtInt is null))
                throw caughtInt;
        }
    }
}
\end{lstlisting}
%TCusedtoendignorethis
This example enables both the use of \texttt{Exception}s within the ATC region and
enables ATC sections to be nested inside each other. However, compared to the
\texttt{Thread} based approach, additional functionality is still required. 
For the safety of the system it must be possible to defer interrupts, enabling 
functions such as memory allocation or system calls to fully complete. In
Real-time Java, this is achieved through the use of runtime reflection to detect
whether the current function is safe to throw an \texttt{Exception} or not.
However, D does not have a concept of run-time reflection, and therefore this cannot be
used. 
Instead, there are two alternatives. For the first, the core runtime and standard
libraries would have to be modified. Through doing this, it is possible to
define a further function property, such as \texttt{@ATCDeferred}.  During these
sections of code, it would not be possible to interrupt. This
would be the preferred method of implementing ATC, as it would remove the
concern over safety from the developer. However, it would require significant
reworking of the D runtime, and possibly compiler alterations (as this would be
a language feature). As such, this method is outside of the scope of this
project. 
The alternative approach is to leave the safety of ATC to the developer by
providing methods to disable and re-enable deferral.
This can easily be achieved through the use of two boolean flags: one to keep track
of whether interrupts are deferred, and one to test whether an interrupt has
arrived while they have been deferred. 
However, as with the \texttt{Thread} approach, an extra method is then needed in 
order to guarantee fully safe execution during nested ATC sections. By keeping
track of each \texttt{Interruptible} section's parent, it is possible to defer
all interrupts, guaranteeing that no \texttt{Error} will be thrown. This is
achieved in the following manner: 
%TCusedtoignorethis
\begin{lstlisting}[basicstyle=\small]
class Interruptible
{
    ..
    void executeSafely(void delegate() fn) 
    {
        defer();
        scope(exit) restore();
        fn();
    }

    private bool previousDeferState;

    private void defer() @safe
    {
        if (! (parent is null) )
        {
            parent.defer;
        }
        previousDeferState = this.deferred;
        this.deferred = true;
    }

    private void restore() @safe
    {
        if (! (parent is null))
        {
            parent.restore;
        }
        this.deferred = previousDeferState;
    }
}
\end{lstlisting}
%TCusedtoendignorethis
This can then be invoked in the same manner as the \texttt{Thread} based
implementation.
\par\bigskip\noindent
Furthermore, as with the implementation of \texttt{Thread} based ATC, a method 
of specifying cleanup is neeeded. This
is possible through the use of D's inbuilt \texttt{scope (exit)}.
This executes code specified on exiting the current scope, regardless of
whether this was through normal flow or as a result of an \texttt{Exception}. 
This provides a convenient method of adding cleanup. 
However, in order to keep the \texttt{Thread} and
\texttt{Exception} based approaches similar in terms of functionality, and to
allow the removal of these cleanup routines, the 
previously \texttt{addCleanup} and \texttt{removeCleanup} functions have
been added to the \texttt{Exception} model. 
\par\bigskip\noindent
As highlighted, there are two possible implementations of Asynchronous
Transfer of Control in D: a two-thread model and a Signal/Exception based
approach. Both of these methods meet all the desired requirements of ATC: 
they can be used safely; allow cleanup; they immediately transfer control; and
they abstract any unnecessary complexity away from the developer. 
However, some aspects of the two approaches differ. 
Using the \texttt{Thread} based approach, a new \texttt{Thread} is
created for each invocation. This suggests a heavier weight implementation compared
to the \texttt{Exception} model.
As far as the cancellation goes, there is little difference between
\texttt{pthread\textunderscore{}kill} and
\texttt{pthread\textunderscore{}cancel}:
both of these functions are operating system calls that use signals in their
implementation. 
However, the \texttt{Exception} based approach to ATC depends heavily on correct 
setup of the target \texttt{Thread's} stack when handling a signal. 
On different CPU architectures, it is possible that the signal handler may not 
allow propagation through the stack.
Therefore, in order to provide a generalisable means of performing ATC, both of
these methods have been implemented. 

\section{Requirement 9: Provision of Testing Facilities}
In order to verify the correct behaviour of the completed implementation, a set 
of tests have been provided. These serve two purposes: they enable verification
of the correct operation of the real-time library; and they improve the
maintainability of the system. 
D provides a built-in 
functionality for unit-testing: using the \texttt{unittest} flag, test code can 
be placed alongside function declarations. In addition, \texttt{assert} statements 
allow expected behaviour to be verified. 
This is shown in the following sample:
\begin{lstlisting}[basicstyle=\small]
int triple(int i) 
{
    return i*3; 
}
unittest 
{
    assert(triple(0) == 0); 
    assert(triple(-4) == -12); 
    assert(triple(10) == 30); 
    assert(12.triple == 36);
}
\end{lstlisting}
When using either of the two most popular compilers, LDC or DMD, a compiler
flag may be used to insert any \texttt{unittest} code into the beginning of the 
main function.
A main function may also be artificially created through a compiler flag:
\texttt{-main}. This allows the full testing of library code through a compiler 
invocation in the following manner: 
\begin{lstlisting}[basicstyle=\small]
ldc -main -unittest -run realtime.d
\end{lstlisting}
This inbuilt definition of \texttt{unittest} allows unit-testing to occur 
without the need for tools external to the core language, thus providing a convenient 
and practical testing method. 
