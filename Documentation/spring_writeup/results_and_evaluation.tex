% Results and Evaluation Section of the Writeup 

% Word Count Guide - 2500 Words
As all the provided code has been thoroughly tested to ensure correct
functionality, this section instead aims to examine the provided library and how its usage
compares syntactically to the development of a real-time system in Ada. 
Following this, Asynchronous Transfer of Control will be profiled in both D and
Ada, comparing the performance of the two. 

\section{Comparison with Ada}
The first notable difference between D and Ada is the syntax of the two languages: 
D follows a C style approach, using curly braces to indicate scope;
Ada, by contrast, follows a Pascal style syntax, using \texttt{begin} and
\texttt{end} statements. However, despite these differences,
the two languages have much in common. 

\subsection{Concurrency}
In Ada, it is possible to separate a concurrent thread of control through the
medium of a \texttt{task}. A \texttt{Task} can be created in the following manner: 
\begin{lstlisting}[basicstyle=\small,language=Ada]
procedure Program is 
    task My_Task; 

    task body My_Task is 
    begin
        ...
    end My_Task;

begin
    null; 
end Program;
\end{lstlisting}
It is worth noting than in the above example, the \texttt{Program} procedure will wait
for \texttt{My\textunderscore{}Task} to complete before exiting. D differs in this aspect: 
there is no concept of ownership of a \texttt{Thread}, and unless
explicitly stated through \texttt{Thread.join}, a process will not await
\texttt{Thread} completion. The notable exception to this is exiting the
\texttt{main} function, which waits for all \texttt{Thread}s to terminate. 
In terms of syntax, the creation of a
\texttt{Thread} in D is significantly different. Rather than specifying the body of
the task inline, it is more typical to place it in a separate function. This
appears as follows: 
\begin{lstlisting}[basicstyle=\small]
import core.thread; 

void threadFunction()
{
    ...
}

void main()
{
    new Thread(&threadFunction).start;
}
\end{lstlisting}
Thus, both languages are able to specify concurrent computation. In
Ada, this is achieved through the medium of a \texttt{Task}, in D, the medium
is provided by a \texttt{Thread}. 

\subsection{Priority Scheduling}
Important to real-time systems is the concept of priority scheduling. In both
Ada and D priority based scheduling is implemented. In Ada, the systems scheduler can
be set through the use of a \texttt{pragma}. This is done as follows: 
\begin{lstlisting}[basicstyle=\small,language=Ada]
pragma task_dispatching_policy(fifo_within_priorities); 
\end{lstlisting}
This allows the priority of a task to be statically set in the following manner: 
\begin{lstlisting}[basicstyle=\small,language=Ada]
task Worker is 
    pragma Priority(System.Priority'First + 1); 
end Worker;

task body Worker is 
begin 
    ...
end Worker; 
\end{lstlisting}
Priority scheduling behaves in a similar manner in D, however, there is one
fundamental difference: the standard approach to priorities in D is not static, 
and any priorities must instead be assigned at runtime, after the
\texttt{Threads} creation. 
Furthermore, priorities can be changed during execution. While similar
behaviour can be implemented in Ada using the
\texttt{Ada.Dynamic\textunderscore{}Priorities}
package, it is the only option for priority scheduling in D. 
Priority scheduling can thus be achieved in the following manner: 
\begin{lstlisting}[basicstyle=\small]
import core.thread; 

void threadFunction()
{
    ...
}

void main()
{
    setFIFOScheduler(Thread.PRIORITY_MAX);
    auto a = new Thread(&threadFunction).start;
    a.priority = Thread.PRIORITY_MIN + 1; 
    Thread.sleep(2.seconds); 
    a.priority = Thread.PRIORITY_MIN + 5; 
}
\end{lstlisting}
This example shows not only an initial assignment of a priority to a created
\texttt{Thread}, but highlights the possibility of modifying a priority at
runtime. 

\subsection{Shared Memory}
With all multithreaded programs, communication between different concurrent
components is crucial. Both Ada and D support the notion of shared memory,
allowing separate \texttt{Threads} or \texttt{Tasks} to access the same
variables. However, due to D's use of Thread Local Storage, shared memory functions
differently. In D, each \texttt{Thread} contains its own copy of a global 
variables unless the variable is stated 
as \texttt{shared} or \texttt{\textunderscore{}\textunderscore{}gshared}. 
\par\bigskip\noindent
In Ada, it is common practice to use protected objects to read and write to
variables, preventing simultaneous access. These operate by defining several different 
operation types: a function allows concurrent reads to occur; a procedure
is exclusive of other procedures and functions, and allows writes to occur; 
an entry allows a task to block on a condition. This is shown
below: 
\begin{lstlisting}[basicstyle=\small,language=Ada]
protected Read_Write is
    function Read return Integer;
    procedure Write(New_Value : in Integer); 
private 
    Data : Integer := 0; 
end Read_Write;

protected body Read_Write is 
begin 
    function Read is 
    body 
        return Data; 
    end Read; 

    procedure Write(New_Value: in Integer) is 
    body
        Data := New_Value; 
    end Write; 
end Read_Write; 
\end{lstlisting}
D does not provide the notion of protected objects, but instead uses a
special \texttt{ReadWriteMutex} that emulates this behaviour.
Using this, the previous Ada code may be rewritten in D
in the following manner: 
\begin{lstlisting}[basicstyle=\small]
auto mut = new ReadWriteMutex(); 
int data = 0; 
void readerFunction()
{
    synchronized(mut.reader)
    {
        return data; 
    }
}

void writerFunction(int newValue)
{
    synchronized(mut.writer)
    {
        data = newValue;
    }
}
\end{lstlisting}
This allows the same principle, protected data, to be used in D for
safe interprocess communication using shared memory. However, the ability to 
define an \texttt{entry} or a \texttt{requeue} is not encapsulated. 
It is worth noting that when using \texttt{ReadWriteMutex}, it is not possible to 
use either the priority ceiling or priority inheritance protocol.
For this to be achieved the \texttt{ReadWriteMutex} class
can be redefined as using the \texttt{CeilingMutex} or
\texttt{InheritanceMutex} classes, rather than a standard \texttt{Mutex}. 

\subsection{Priority Inheritance and Priority Ceiling Protocols}
As detailed in the Literature Review, the ability to use either the priority
inheritance protocol, or the priority ceiling protocol is crucial in order to
provide a bounded limit on the about of priority inversion that a task can
suffer. Ada supports use of the priority ceiling protocol in the following
manner: 
\begin{lstlisting}[basicstyle=\small,language=Ada]
protected Buffer is 
    pragma Priority(28); 
    procedure accessResource; 
    ...
end Buffer; 
\end{lstlisting}
When accessing the Buffer, the \texttt{Task} has its priority raised
to 28. 
In D, the same functionality is implemented through 
the \texttt{CeilingMutex} class: an accessing \texttt{Thread} 
\texttt{CeilingMutex} has its priority raised as it locks the mutex. 
The D equivalent appears in the following manner: 
\begin{lstlisting}[basicstyle=\small]
auto mut = new CeilingMutex(28); 

void accessResource()
{
    synchronized(mut)
    {
        ...
    }
}
\end{lstlisting}
As with the implementation of priorities, the ceiling property in D is dynamic 
and can be adjusted at runtime. 
Unlike Ada, this project also supports the use of priority inheritance. 
This can be similarly invoked in the following manner: 
\begin{lstlisting}[basicstyle=\small]
auto mut = new InheritanceMutex; 

void accessResource()
{
    synchronized(mut)
    {
        ...
    }
}
\end{lstlisting}
Through these examples, it is shown that D provides two alternative protocols
to solve priority inversion: the priority ceiling protocol, which behaves the
same as Ada's priority ceiling protocol, and the priority inheritance protocol. 

\subsection{Periodic Tasks}
For a real-time system, periodic tasks are a common feature. In Ada, a periodic
task is defined in the following manner: 
\begin{lstlisting}[basicstyle=\small,language=Ada]
task Periodic; 
task Periodic is 
    Release_Interval : Time_Span := Milliseconds(100); 
    Next_Release : Time := Clock + Release_Interval; 
begin
    loop 
        delay until Next_Release; 
        ... 
        Next_Release := Next_Release + Release_Interval; 
    end loop; 
end Periodic; 
\end{lstlisting}
This can be defined almost identically in D using the provided \texttt{delayUntil} function: 
\begin{lstlisting}[basicstyle=\small][language=Java]
void periodic()
{
    immutable releaseInterval = 100.msecs; 
    auto nextRelease = MonoTime.currTime + releaseInterval; 
    while(true)
    {
        delayUntil(nextRelease); 
        ...
        nextRelease += releaseInterval; 
    }
}

void main()
{
    new Thread(&periodic).start; 
    ...
}
\end{lstlisting}
This therefore allows a periodic task to be defined in a manner 
similar to that of Ada. There is little visible difference between the two
implementations, and as both utilise operating system threads, there is little
difference in overhead between the two languages. 

\subsection{Asynchronous Transfer of Control}
Another crucial aspect of a real-time system is the ability to perform
Asynchronous Transfer of Control (ATC). In Ada, this is achieved by the 
\texttt{select ... then abort} statement. 
For example, a timeout of 2 seconds on a function call takes the 
following form: 
\begin{lstlisting}[basicstyle=\small,language=Ada]
select 
    delay 2.0; 
then abort
    abortable_function; 
end select; 
\end{lstlisting}
In D, it is not possible to program a timeout in this manner: the method of ATC
is fundamentally very different, taking instead a class based approach using 
signals. An \texttt{Interruptible} task is defined, this enables
execution of a function to be performed until there is a call to
\texttt{interrupt}. For programming a timeout in D an additional
thread has to be created compared to the Ada implementation. 
This appears as follows: 
\begin{lstlisting}[basicstyle=\small]
void abortableFunction()
{
    ...
}

void main()
{
    Interruptible intr = new Interruptible(&abortableFunction); 
    new Thread({
        Thread.sleep(2.seconds); 
        intr.interrupt(); 
    }).start;
    intr.start; 
}
\end{lstlisting}
This allows a implementation of ATC in D, sharing many similarities with the
behaviour of Ada. However, in some regards it is subtly different to 
Ada's model. The most significant difference is the management of
whether a section of code is abort deferred or not. In Ada, there is notion of
a task being abort-deferred, specified by \texttt{pragma Abort\textunderscore{}Defer}
\cite{atc-article}.
This prevents asynchronous cancellation for
critical regions of code, so that they may safely perform operations such as
updating shared memory or allocating memory. 
For most standard interactions, such as with protected objects or standard
packages, there is no need to explicitly specify code as abort-deferred. This
therefore removes the concern from the developer. In D, there is no notion of a
function being abort-deferred. Instead, as a result of having an unchanged
compiler and runtime, it is left up to the developer to
control whether an ATC region of code may have interrupts deferred, using the 
\texttt{Interruptible.getThis.deferred} flag. 

\section{Comparison of Asynchronous Transfer of Control Approaches in D}
% Profile the two ATC Methods. 
Two alternative approaches to providing ATC
have been implemented in D. In this section, both of these implementations will 
be profiled, comparing the performance benefits of a one thread, exception approach over a 
two thread model. 
\par\bigskip\noindent
For profiling, a sample program was created that records the
time taken for both the setup of an \texttt{Interruptible} section, and the time 
taken to progress from receiving an asynchronous \texttt{interrupt}. In order to 
provide reliable results,
this was repeated 10,000 times with average times and standard deviations recorded. 
Furthermore, both
the \texttt{Thread} model, and the \texttt{Exception} model have been compiled 
using the same version of the DMD compiler (2.069.2), with no compiler
optimisation. This was run on a Linux system at a high
priority (90) in order to minimise the amount of interference that the
profiling might suffer. 
The D code for profiling the Exception or Thread based ATC is as follows: 
\lstinputlisting[basicstyle=\small,language=C++]{exception_profile.d}
These tests produced the following results for the setup and cancellation times: 
\begin{table}[!htbp]
\begin{tabular}{l|lllll}
    ATC Method & Average            & Setup Standard           & Cancellation     & Cancellation       \\
               & Setup time         & Deviation                & Time             & Standard Deviation \\ 
               & (microseconds)     & (microseconds)           & (microseconds)   & (microseconds)     \\ \hline
    Threaded   & 38.05              & 12.31                    & 42.09            & 14.77              \\
    Exception  &  4.04              & 1.84                     & 15.02            & 11.57              \\
\end{tabular}
\end{table} \\
% Ada times & 1.11              & 69.11          & 70.22  \\ 
As expected, this shows that there is an additional overhead in the
\texttt{Thread} based approach to ATC. This occurs as the creation of a 
\texttt{Thread} in D involves the creation of a kernel level thread, which is a 
heavier weight process than the creation of an \texttt{Exception}.
Furthermore, the \texttt{Thread} based approach also takes longer to perform
the cancellation. Both approaches use a POSIX signal in order to asynchronously
perform their cancellation: the \texttt{Exception} based approach uses a signal
to throw an exception and the \texttt{Thread} based approach uses a signal to
terminate the Thread. However, with the \texttt{Thread} based approach, the
application must wait for the \texttt{Thread} to be fully completed before it can
continue. This is a lengthier process than catching an
\texttt{Exception}, as resources must be reclaimed and any \texttt{Thread}
cleanup performed. 

\section{Efficiency of ATC compared to Ada}
In addition to comparing the two ATC mechanisms, a comparison against Ada can be 
used to highlight the performance of D. As in the
previous example, it is possible to profile both the setup time and cancellation
time of Ada's ATC. This takes the following form: 
\lstinputlisting[basicstyle=\small,language=Ada]{./ada_profile.adb}
A profiling of the Ada implementation of ATC gives the following results: 
\begin{table}[!htbp]
\begin{tabular}{l|llll}
ATC Method & Average            & Setup Standard           & Cancellation     & Cancellation       \\
           & Setup Time         & Deviation                & Time             & Standard Deviation \\ 
           & (microseconds)     & (microseconds)           & (microseconds)   & (microseconds)     \\ \hline
D Threaded   & 38.05            & 12.31                    & 42.09            & 14.77              \\
D Exception  &  4.04            & 1.84                     & 15.02            & 11.57              \\
Ada        & 1.39               & 1.50                     & 76.27            & 10.24               \\
\end{tabular}
\end{table} \\
This shows a different distribution of setup and cancellation times compared to
the previous D profiles: Ada has almost no setup cost, but a much greater
cancellation time. Using \texttt{strace} on the gnatmake compiled Ada program 
shows that a two-thread approach is taken. 
However, the thread created is much more lightweight than a D \texttt{thread}: it shares memory
with the parent process. Additionally, Ada takes a safer, although less immediate 
approach towards aborting: Ada has abort completion points. This prevents ATC 
occuring except at locations in the code that are known to be safe. This
approach trades abortion for safety. As the implementation in D contrastingly cancels 
immediately, it is the reason for the difference between cancellation times. 

