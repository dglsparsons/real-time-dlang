% Results and Evaluation Section of the Writeup 

% Word Count Guide - 2500 Words
This section sets out to examine the provided library, and how its usage
compares syntactically to the development of a real-time system in Ada. 
Following on from
this, a comparison in terms of efficiency will be made with Ada. Furthermore,
the efficiency of the two implementations of Asynchronous Transfer of Control in 
D will be examined. 

% Appreciate the latent issues of the subject area (for example, 
% in software engineering they might meet and tackle such as emergent requirements, 
% design flaws, equipment/application problems). Examined in the project 
% design/implementation and evaluation chapters. 
\section{Comparison with Ada}

This section sets to compare the implementation of real-time primitives in D,
with those of Ada, a general purpose programming language with a wide range of
support for real-time systems. The syntax of the two languages is significantly 
different: D follows a C style approach, using curly braces to indicate scope,
Ada by contrast follows a Pascal style syntax, using \texttt{begin} and
\texttt{end} statements to indicate scope. However, despite these differences,
the two languages have much in common. 

\subsection{Concurrency}

In Ada, it is possible to separate a concurrent thread of control through the
medium of a task. This enables a level of parallelism. A \texttt{Task} can be 
created in the following manner: 
%TC:ignore
\begin{lstlisting}[language=Ada]
procedure Program is 
    task My_Task; 

    task body My_Task is 
    begin
        ...
    end My_Task;

begin
    null; 
end Program;
\end{lstlisting}
%TC:endignore
It is worth noting than in the above example, the \texttt{Program} procedure will wait
for \texttt{My\textunderscore{}Task} to complete before exiting. D differs in this aspect: 
there is no concept of ownership of a \texttt{Thread}, and unless
explicitly stated through \texttt{Thread.join}, a process will not await
\texttt{Thread} completion. The notable exception to this is exiting the
\texttt{main} function, which waits for all threads to terminate. 
In terms of syntax, the creation of a
\texttt{Thread} in D is similarly different. Rather than specifying the body of
the task inline, it is more typical to place it in a separate function. This
appears as follows: 
%TC:ignore
\begin{lstlisting}
import core.thread; 

void threadFunction()
{
    ...
}

void main()
{
    new Thread(&threadFunction).start;
}
\end{lstlisting}
%TC:endignore
Thus, both languages are able to similarly specify concurrent computation. In
Ada, this is achieved through the medium of a \texttt{Task}, in D, the medium
is provided by a \texttt{Thread}. 

\subsection{Priority Scheduling}
Important to real-time systems is the concept of priority scheduling. In both
Ada and D priority based scheduling is implemented. In Ada, the systems scheduler can
be set through the use of a \texttt{pragma}. This is done as follows: 
%TC:ignore
\begin{lstlisting}[language=Ada]
pragma task_dispatching_policy(fifo_within_priorities); 
\end{lstlisting}
%TC:endignore
This allows the priority of a task to be statically set in the following manner: 
%TC:ignore
\begin{lstlisting}[language=Ada]
task Worker is 
    pragma Priority(System.Priority'First + 1); 
end Worker;

task body Worker is 
begin 
    ...
end Worker; 
\end{lstlisting}
%TC:endignore
Priority scheduling behaves in a similar manner in D, however, there is one
fundamental difference: priorities in D are not static, and must instead be
assigend at runtime. Furthermore, priorities can be changed throughout 

\subsection{Shared Memory}

\subsection{Priority Inheritance and Priority Ceiling Protocols}

\subsection{Periodic Tasks}

\subsection{Sporadic Tasks}

\subsection{Asynchronous Transfer of Control}

\section{Efficiency of Implementation vs Ada}
% Compare D with Ada in terms of efficiency.
In order to form a true comparison between the two languages, D and Ada, it is
necessary to examine their performance.

\section{Comparison of Asynchronous Transfer of Control Approaches}
% Profile the two ATC Methods. 
