% Conclusion section to the report. 

% Word Count - 1000 words

The D programming language provides a capable set of features for the 
development of real-time systems: the ability to interact with low-level 
hardware, powerful concurrency tools, and interoperability with existing C code. 
However, these alone are not enough to provide the full tool-set required for
the development of a real-time system. 
Through the course of this project, these additional requirements have been 
explored, discussed, and implemented.
\par\bigskip\noindent
By targeting POSIX compliant operating systems, particularly those that are
real-time operating systems, it is possible to provide tools that enable control 
over the operating system's scheduler. 
This allows priority based scheduling to be used, thus satisfying requirements
1 and 2.
In order to have high precision timings, access to a monotonic clock is provided 
as a language feature. 
This clock can be then used in combination with C
interoperability to provide absolute delay timings through a wrapper around the
\texttt{clock\textunderscore{}nanosleep} function, satisfying requirements 3 and
4. 
In order to provide a bound on the amount of priority inversion that a task
may suffer, both the Priority Inheritance and Priority Ceiling protocols have
been implemented through the languages use of POSIX \texttt{Mutexes}. This meets
requirements 5 and 6. 
For requirement 7, Asynchronous Transfer of Control (ATC) has been implemented in two 
alternative approaches: a two-thread model, and a one-thread \texttt{Exception} model. 
This provides the desired functionality and 
nicely encapsulates not only creation and cancellation of an
abortable section, but provides mechanisms for adding cleanup routines and deferring 
cancellation. 
\par\bigskip\noindent
In the Results and Evaluation section, the two alternative implementations of
ATC in D have been profiled. 
This found that, while an \texttt{Exception} based approach depends 
heavily on the setup of the stack during signal handling, it is a much more efficient
implementation than the alternative two-thread approach. 
Additionally, the approaches to ATC in D have been profiled against an
equivalent Ada program, highlighting that both systems have similar performance.
This displays that with the given approach, making no changes to D's runtime or
compiler, it is possible to achieve all the functionality required. Most
features are implemented safely through a binding onto real-time POSIX.
However, the approach taken to ATC is limited by not changing the compiler and
runtime: deferred code has to be explicitly stated, unlike in Ada's implementation. 
\par\bigskip\noindent
Furthermore, in addition to the provided functionality, a wide range of unit-tests 
have been provided, along with in-built documentation. 
These unit-tests can be invoked to ensure correct operation on a specific
system, and provide example usage of the introduced features. 
All the implemented tools are provided as a module, and can be made available as 
a library. This would enable the development of real-time system in the D programming 
language, as well as maintaining compatibility with existing code. 
\par\bigskip\noindent
However, there are additional difficulties that the 
language faces that could provide avenues for future work. 
First, the use of a stop-the-world garbage collector prevents a reliable 
estimate of timings from being determined. Future work could replace this with a
real-time garbage collector. 
Secondly, as D's standard libraries are so expansive, there is also potential for future
work in making the standard library fully real-time compliant: features such as
\texttt{Tasks} and \texttt{Fibers}, which can be used to execute work
concurrently, currently do not support the use of priority based scheduling.
Finally, further work could port the provided library to a non-POSIX compliant
real-time operating system. For the majority of features, a wrapper could
be provided around the operating system calls. For ATC either thread cancellation 
or signals must be present. 
\par\bigskip\noindent
Therefore, it is possible to develop a real-time system in the D programming 
language using the additional support detailed in this project. This is
significant due to the rich set of features D provides compared to other
real-time systems programming languages. By providing modern primitives such as
unit-testing as language features, but also boasting native performance, 
memory safety and powerful abstractions, D provides a modern approach to the
development of real-time systems. 
