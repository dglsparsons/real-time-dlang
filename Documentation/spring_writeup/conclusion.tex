% Conclusion section to the report. 

% Word Count - 1000 words

The D programming language provides a capable set of features for the 
development of real-time systems, such as: the ability to interact with low-level 
hardware, powerful concurrency tools, and the ability to interact with existing C code. 
However, these alone are not enough to provide the full tool-set required for
the development of a real-time system. 
As such, additional requirements have been explored, discussed, and implemented 
through this project. 
\par\bigskip\noindent
By targeting POSIX compliant operating systems, particularly those that are
real-time operating systems, it is possible to provide tools that enable control 
over the operating system's scheduler. 
This allows priority based scheduling to be used, thus satisfying requirements
1 and 2.
In order to have high precision timings, access to a monotonic clock is provided 
as a language feature. 
This clock can be then used in combination with C
interoperability to provide absolute delay timings through a wrapper around the
\texttt{clock\textunderscore{}nanosleep} function, satisfying requirements 3 and
4. 
In order to provide a bound on the amount of Priority Inversion that a task
may suffer, both the Priority Inheritance and Priority Ceiling protocols have
been implemented through the languages use of POSIX \texttt{Mutexes}. This meets
requirements 5 and 6. 
For requirement 7, Asynchronous Transfer of Control has been implemented in two 
alternative approaches: a two-thread model, and a one-thread \texttt{Exception} model. 
This provides the desired functionality and 
nicely encapsulates not only creation and cancellation of an
abortable section, but provides mechanisms for adding cleanup routines and deferring 
cancellation. 
\par\bigskip\noindent
In the Results and Evaluation section, the two alternative implementations of
Asynchronous Transfer of Control in D have been profiled. 
This found that, while an \texttt{Exception} based approach depends 
heavily on the setup of the stack during signal handling, it is a much more efficient
implementation, taking around only 40\% of the computation time compared to a
two-thread approach. 
Additionally, the approaches to ATC in D have been profiled against an
equivalent Ada program, highlighting that both systems have similar performance.
\par\bigskip\noindent
Furthermore, in addition to the provided functionality, a wide range of unit-tests 
have been provided, along with in-built documentation. 
These unit-tests can be invoked to ensure correct operation on a specific
system, and provide example usage of the introduced features. 
All the implemented tools are provided as a module, and can be made available as 
a library. This would enable the development of real-time system in the D programming 
language, as well as maintaining compatibility with existing code. 
\par\bigskip\noindent
However, there are additional difficulties that the 
language faces: the use of a stop-the-world garbage collector prevents a reliable 
estimate of timings from being determined. 
This is a problem that can be offset through disabling the garbage collector 
and instead using manual memory management, or through detaching a thread from the 
runtime and ensuring any GC objects used are referenced from other threads.
Alternatively a real-time garbage collector could replace the existing one. 
Offsetting the penalty of the garbage collector adds additional complexity for 
the programmer and can restrict the use of more advanced language features.
This opens an avenue for potential future work in this field. 
As D's standard libraries are so expansive, there is also potential for future
work in making the standard library fully real-time compliant: features such as
\texttt{Tasks} and \texttt{Fibers}, which can be used to execute work
concurrently, currently do not support the use of priority based scheduling. 
\par\bigskip\noindent
As such, it is possible to develop a real-time system in the D programming 
language using the additional support detailed in this article. However, 
the Garbage collector remains a significant hurdle in aiding development.
%TODO - Make out why you would use D over some other programming languages for
%the development of real-time systems. 
% I.E it's better than C++, which is used on fighter jet, because it provides a
% neater encapsulation, advanced features, built in unit-testing, and more
% robust safety.  POINT - Evidence, evaluation and more evaluation. 
