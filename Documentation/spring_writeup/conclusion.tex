
The D programming language inherently provides a capable set of features for the 
development of real-time systems, such as the ability to interact with low-level 
hardware, powerful concurrency tools, an interesting approach to memory 
safety, and the ability to interact with existing C code. 
However, these alone are not enough to provide the full tool-set required. As such, 
additional features have been discussed and implemented. 
By targeting POSIX compliant operating systems, it is possible 
to provide tools that enable control over the operating systems scheduler, 
allowing priority based scheduling to be used. 
Through the 
language's use of POSIX mutexs, it is also possible to prevent unbounded 
priority inversion from occurring.
In order to have high precision timings, access to a monotonic clock is provided 
as a language feature. This can be used to provide absolute delay timings. 
Asynchronous Transfer of Control has provided a difficult feature to design and 
implement in a reliable manner, but can be implemented in X way.
\par\bigskip\noindent
The mentioned tools can be implemented in a module, and hence made available as 
a library. 
This would enable the development of real-time systems in the D programming 
language, as well as maintaining compatibility with existing code. 
\par\bigskip\noindent
However, there are additional difficulties that the 
language faces: the use of a stop-the-world garbage collector in the 
D programming language prevents a reliable estimate of timings from being determined. 
This is a problem that can be offset through disabling the garbage collector 
and instead using manual memory management, or through detaching a thread from the 
runtime and ensuring any GC objects used are referenced from other threads.
Offsetting the penalty of the garbage collector adds additional complexity for 
the programmer and can restrict the use of more advanced language features. 
\par\bigskip\noindent
As such, it is possible to develop a real-time system in the D programming 
language using the additional support detailed in this article. However, 
the Garbage collector remains a significant hurdle in aiding development.
