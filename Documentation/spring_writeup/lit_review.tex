% Literature Review for my project. 

% Word Count Guide - 3000 words

% One or more review chapters, describing the research I did at the beginning
% of the project period. 

% Demonstrate that they have acquired specialisation in a particular part of the
% subject area, including enhanced or new technical skills that build on taught
% theory.  Examined in the overall project report. 

In this chapter, the literature surrounding real-time systems will be
reviewed. The various constructs that enable the development of a real-time
system will also be examined.

\section{Features of a Real-Time System}

A significant amount of research has gone into the addition of features to
real-time languages. Through these, a multi-processor system can be
programmed in such a way that all tasks will operate within a finite and 
predictable timespan. Furthermore, these allow precise timings to be
used, ensuring that deadlines within a system can be reliably met.  Burns and
Wellings provide an overview of development for real-time systems in
their book, \emph{Real Time Systems and Programming Languages}
\cite{real-time-systems}.  This provides a comprehensive set of
information on: 
the base requirements of a real-time system, 
the implementation of real-time systems in various programming languages, 
and all relevant theory surrounding the field. 
In providing a detailed overview of real-time programming concepts, 
this book sets a background the implementation required for
this project.

\subsection{Concurrent Programming}
% Dijkstra
Ever since Dijkstra introduced the concepts of 
mutual exclusion, processes, semaphores, deadlock, 
preconditions and guards in his 1968 seminal paper,
\emph{Cooperating Sequential Processes}, concurrent programming has seen an
explosion in popularity \cite{Dijkstra}.  Synchronisation and
concurrency have become fundamental issues of Computer Science, with a 
range of research exploring the field's potential. 
Furthermore, this research has provided mechanisms that aid 
the development of concurrent programs
\cite{Sutter:2005:SCR:1095408.1095421,Hansen:1972:SM:361454.361473}, notably
the introduction of monitors by Tony Hoare.
These enable a simplification of logic and are used in the introduction of many
abstractions such as Ada's protected objects
\cite{Hoare:1974:MOS:355620.361161}.  
\par\bigskip\noindent
% Then apply the concurrent aspect of this to a real-time system! 
Real-time systems benefit from parallelism. Inherently, they are
concerned with the notion of timing and scheduling many different components
across a system.  The ability to schedule these components concurrently and in
isolation thus provides a valuable tool. Concurrent programming 
enables complex systems to be more simply developed, and provides 
more readable, modular systems. Furthermore, concurrency reduces the complexity
of scheduling these systems \cite{real-time-systems}.  It is therefore a
requirement of a real-time systems programming language to provide support for
an abstraction into isolating concurrent events and their control into separate
entities.  In Java and C with real-time POSIX, this concurrency mechanism is
provided by threads or processes. In Ada, this mechanism is provided by tasks
\cite[p251]{gehani1989concurrent}.

\subsection{Priority Scheduling}
%Some theory on priority based scheduling.
In addition to the separation of concerns through
concurrent programming, the ability to schedule system components in a
predictable manner is of interest. Much research has gone into the development
of fair scheduling algorithms, such as those of the Linux
scheduler \cite{6043273,7280991}. However, these do not provide guarantees on
the scheduler's behaviour, and tasks with short deadlines may have to wait for
extended periods of time. Contrastingly, real-time systems do not care about the
scheduler implementing a fair algorithm. Scheduling policies have been
developed that focus on ensuring short deadline tasks are executed in
preference to others, enabling them to meet their deadlines.
% Apply it to real-time systems
\par\bigskip\noindent
Many algorithms exist that enable real-time systems to be
scheduled, such as Earliest Deadline First (EDF), Least Laxity First (LLF), or
Fixed Priority scheduling \cite{real-time-systems}.  However, each policy
requires separate computational analysis to determine whether the
system will meet all its deadlines. The field of schedulability analysis is
based around this, with research aiming to provide tests which determine if a
system will behave as expected \cite{4815215,burns-sched-analysis}.  Even though 
Burns provides a test for proving the schedulability of EDF systems, these systems 
are more difficult to implement in practice as priorities are dynamic, and may change
at any time. Furthermore, not all tasks have a deadline, making some
difficult to incorporate. As a result, the most commonly used scheme is Fixed 
Priority scheduling. 
\par\bigskip\noindent
It is therefore a requirement of a real-time systems programming language to provide
support for a priority based scheduling algorithm or allowing the scheduler to
be changed. 

\subsection{Bounded Priority Inversion}
% Talk about the priority inversion problem. Introduce it.  Talk about existing
% solutions to the priority inversion problem - Priority Ceiling protocol, and
% the Priority Inheritance protocol. % Show the research that they come from.
Concurrent components of a system are rarely isolated,
often requiring some shared information or data.
Using priority based scheduling, it is possible for a high priority
task to become blocked from making progress due to a low priority task holding
a resource. Additionally, the low priority task may be preempted and prevented
from making progress. In this instance, the high priority task is effectively
running at the lowest priority of the system: it must wait for all other
tasks to complete before it can progress. This is a well studied phenomenon
known as priority inversion \cite{real-time-systems}.  
\par\bigskip\noindent
In
order to provide a reliable response in a priority based system, it is
therefore necessary that priority inversion is bounded.  Two main protocols
exist that solve this problem: the Priority Inheritance protocol and the
Priority Ceiling protocol. Rajkumar and Lehoczky define these protocols in their
paper \cite{57058}. The Priority Inheritance
protocol works by raising the priority of a task holding a resource to the
priority of any task that attempts to gain access to the resource.
Contrastingly, the Priority Ceiling protocol raises the priority of the task as
it first accesses the resource to the highest priority of any task that may
access the resource.  Each protocol has its own respective strengths and
weaknesses, but enables reliable behaviour of the system \cite{mall2009real}.
It is therefore a requirement of a real-time systems programming language that
the underlying data sharing system is not subject to unbounded priority
inversion: the ability to specify the use of either the Priority Inheritance
protocol or the Priority Ceiling protocol is a necessary feature.  

\subsection{Monotonic Clocks and Absolute Sleep}
% Talk about studies of clocks, and drift for real-time systems. 
Up to this point, it has been assumed that real-time systems are able to
interact with precise timing requirements to a fine degree of accuracy.  
In practice, this is not always the case, in particular with the use
of distributed systems. There is a large field of research surrounding the
synchronisation of multiple devices, with algorithms such as Cristian's
algorithm for synchronising systems \cite{37958}. However, the crucial aspect is 
access to a monotonic clock: a clock that ticks at a constant rate and does
not vary. Baker and Pazy in their 1991 paper,\emph{ Real-time Features for Ada
9X},
cite the need for high precision, non-changing clocks as a language feature \cite{160371}.  
It is therefore clear that it is a requirement of a real-time systems
programming language to provide access to monotonic clocks, so high
precision timing requirements can be met.  
\par\bigskip\noindent
Additionally,
the ability to awaken threads at times stated by a clock is a requirement of
real-time systems. A typical \texttt{sleep} function will cause the calling
thread to hand over any computation resources for a defined period of time.  In
a real-time system, the timing that the thread awakens must be precise.  The
typical approach involves the use of a relative sleep time, and therefore can
be imprecise due to preemption \cite{real-time-systems}: calculating the sleep
time and then being preempted before the call to \texttt{sleep} will lead to
incorrect timings. To achieve a predictable sleep function, it is fundamental 
to have the ability to wait until an absolute time using a high precision clock, 
such as a monotonic
clock.

\subsection{Asynchronous Transfer of Control}
% Introduction to ATC 
Asynchronous Transfer of Control (ATC) is a controversial topic due to
its implementation in many languages, yet it is deemed a useful, if not
necessary, feature of a real-time system \cite{atc-article}.  ATC is the
transfer of control within a thread, triggered by an external
event or thread.  Its effect may be instantaneous or deferred until a safe
section in the code has been reached. In addition, sections of code susceptible
to ATC may be nested, causing a complex problem in terms of semantics,
methodology, and implementation. The use of ATC extends to many
purposes including: low latency responses to events, timing out a thread of
computation, or terminating threads, making it a valuable feature. 
% Have 2-3 articles on the different approaches to ATC within real-time systems
% programming languages
\par\bigskip\noindent
The various methods of encapsulating ATC have been well studied, with 
implementations existing in Java
and Ada. Brosgol and Wellings detail the differences between the Ada and
Real-time Java implementations: Java favours a class based approach to
the protocol, where Ada instead provides an explicit language syntax
\cite{atc-article}. Both solutions favor safety of the system and do not
necessarily provide immediate cancellation. Brosgol 
details that there are two alternative implementations for
ATC \cite{Brosgol:2002:ATC}.
The first is achieved through aborting a thread, in which there is a notion of
an abort-deferred region. 
This method avoids much of the complexity of ATC, including the ability to
propagate cancellation, but it has the additional overhead of thread management and
thread dependencies.  
The second approach discussed is an exception based model. 
In this, ATC is captured by allowing a thread to ``arbitrarily throw an
exception at a target thread'' \cite{Brosgol:2002:ATC}.  
Although this method avoids the overhead of additional
thread creation, it requires additional safety to prevent exceptions
from being thrown when the target thread is not able to handle them.
\par\bigskip\noindent
It is therefore evident that ATC
is fundamental for a real-time system. However, the implementation of 
this principle requires careful consideration. 
To maintain a safe system, threads may not be terminated at time: they must have 
the ability to defer interrupts when desired. 
Additionally, there are two main methods of
implementation: a thread cancellation based model and an asynchronous
exception model. While the thread model has additional overhead due to the
creation of new threads, the exception based approach has additional complexity
requiring the ability to safely defer cancellation.

\section{Abstractions}

Combined with the fundamental ability for a programming language to provide
support for real-time systems programming principles, the language must provide
abstractions to ease development.

\subsection{Periodic Tasks} 
Ada does not provide a mechanism for
abstracting periodic tasks, and relies instead on their manual construction. 
By contrast, Real-time Java directly provides mechanisms for easily creating a periodic task
through the \texttt{PeriodicParameters} class \cite{real-time-systems}. 
Bruno  introduces the concept of a periodic task
and explains how it might be implemented using the Java defined
structures \cite{periodic-java-thread}.
Similarly, Burns and Wellings 
demonstrate how a periodic task can be manually configured in
Ada through the use of the \texttt{delay until} statement \cite{burns1998concurrency}.
\par\bigskip\noindent
It is therefore possible to see that both Ada and
Real-Time Java provide a mechanism for the programming of periodic tasks. Java
achieves this through the abstraction of the \texttt{PeriodicParameters} and
\texttt{PeriodicThread} classes.  These allow the programmer to adjust the 
parameters as needed.  Through a more low-level
approach, Ada provides a \texttt{delay until}
statement. This provides the basic building block for a periodic task.

\subsection{Sporadic Tasks}
In addition to providing tools that support the development of periodic tasks,
both Real-time Java and Ada have implementations that permit the programming of
sporadic tasks.  Sporadic tasks are defined as a task that may be
released at any time and are typically event triggered. However,
there is a limit on the rate at which a sporadic task may occur. This is
typically specified through a minimum time between releases \cite{11111101}.
Using the Ravenscar-Java profile described in \cite{ravenscar-java}, a
predefined class, \texttt{SporadicEventHandler} is detailed. Schoeberl shows
how this may be used to achieve an event-triggered computation in an embedded
system using Real-time Java with the Ravenscar-Java profile \cite{1300334}.
\par\bigskip\noindent
Ada takes a similar approach to that which it uses for periodic tasks. Burns
and Wellings detail the potential implementation in \emph{Real-Time Systems and
Their Programming Languages} \cite[p341]{real-time-systems}.  This approach
uses a protected object to handle an interrupt that releases a task.
As a result, it is clear to see that both Ada and Real-Time Java have
implementations allowing a sporadic task to be programmed. 

\section{Language Design} 
Programming languages have also been a major topic of study in Computer 
Science. Significant research has gone into providing
capable, efficient and readable programming languages. This project
does not aim to compare the merits of programming language research. However, the
feature and literature surrounding aspects of programming languages of
interest for this project have been highlighted. 

\subsection{Interprocess Communication and Synchronisation} % 200 words 
As previously mentioned, sharing memory between processes has long been an
important consideration. Hoare's concepts of monitors, coupled with the use of semaphores
and mutexes, is a common approach for sharing information between threads
\cite{Hoare:1974:MOS:355620.361161}.  However, message passing provides an
alternative method of interprocess communication, relying on message boxes for
each task, rather than sharing memory. The use of
message passing in languages such as Erlang is well studied, with
Erlang being renowned for its concurrency capabilities. This is highlighted by
Vinoski in his study of the Erlang message passing system \cite{6216341}. 
A similar approach is taken in Ada's model of
concurrency. Ada uses a rendezvous, a special form of synchronous
message-passing, to communicate between tasks. This allows
messages to be passed without locks and mutexes. However, unlike
Erlang, Ada also implements shared memory. This allows the use of semaphores
and mutexes to communicate between processes
\cite{burns1998concurrency}.  
\par\bigskip\noindent
Esterel is another relevant language due to its unconventional approach to 
concurrency. Rather than focusing on concurrent execution of system components, 
Esterel provides a synchronous approach, assuming that all calculation 
takes zero time. This means that all actions are instantaneous and atomic, 
and therefore two actions cannot interfere with each other. 
Beneficially, this enables the system to be fully deterministic. 
Furthermore, it removes the concern over concurrent languages such as Ada, 
as Esterel does not force the user to ``choose between determinism and concurrency''.
However, due to the language's current status, in development since the 1980s, 
it has yet to see widespread adoption \cite{esterel}.
\par\bigskip\noindent
In summary, language design
may seek to implement various approaches for the communication of data, 
the most significant of which are message passing and shared memory. 
Some programming languages, such as Ada, implement both of these.

\subsection{Atomic Operations} % 200 words 
Rather than solely controlling access to shared memory, many languages also have support 
for atomic operations. An atomic operation is an action that is
indivisible and happens in an uninterruptible manner
\cite{preshing}. Many languages implement these features as part of their
standard libraries, as seen in Java and C \cite{java-atomic,preshing}.  
\par\bigskip\noindent
As atomic operations are indivisible, they
allow concurrent reading and writing of shared data without the need for
protection against concurrent access. Goetz has analysed the capabilities of
Java using atomic operations, concluding that atomic access to variables can 
provide a ``high-performance'' replacement for
many control structures that typically require synchronization
\cite{java-ibm-atomic}.  
\par\bigskip\noindent
The use of atomic operations in real-time systems is relevant.  
A common issue in real-time systems is the need to consider the blocking a task
might suffer when accessing a shared resource. Using atomic operations, tasks 
do not block when waiting to access a shared memory location. Huang, Pillai and 
Shin demonstrate the importance of non-blocking and wait free approaches to 
synchronization in their paper \cite{Huang:2002:IWA:647057.713863}.
Atomic actions enable a real-time system to have high-performance
access to shared data, and ensure that the system does not suffer from 
priority inversion. The inclusion of atomic actions in a programming 
language is therefore desirable for a real-time system. 

\subsection{Exception and Error Handling} % 200 words
In real-time systems, the ability to handle errors caused through execution of
the system, and the ability to affect the normal flow of control is of
interest. Programming languages offer various different implementations of
error or exception handling, but the most commonly used approach is a 
\texttt{try\{\} catch\{\} } method. Kiniry has studied the implementation in 
Eiffel and Java, citing these as the ``extremes in design and application''. 
Kiniry states that most modern programming languages can be divided into two
groups based on whether the exceptions are typically used as flow control
structures or to handle abnormal situations
\cite{Kiniry:2006:EJE:2124243.2124264}. 
\par\bigskip\noindent
Despite the widespread use of exception semantics in programming languages,
they have received criticism. Weimer and Necula's studies revealed
that the exception semantics are a common cause of errors, with error handling
often being ``quite labyrinthine'' \cite{Weimer:2008:ESP:1330017.1330019}. 
Similarly, Hoare has criticised Ada's use of exceptions, referring to them as
``dangerous'' \cite{Hoare:1981:EOC:358549.358561}. Despite this,
exception handling a popular mechanism for fault handling, and their
use in Ada and Real-time Java shows the viability for real-time systems.

\subsection{Low Level Capabilities} % 200 words
Next is the ability to handle low level concerns without unnecessary abstractions. 
Generally this refers to the ability to use assembler or manually manipulate hardware. For a real-time
system, the concern is primarily with interrupts and IO operations. With many
modern languages running in a virtual machine environment, access to hardware
is limited. Significant effort has gone into providing
Real-time Java with support for low-level access \cite{4519616,real-java}. 
Comparatively, C and Ada do not run in a virtual machine and therefore
potentially have direct access to interrupt handlers \cite{real-time-systems}.

\subsection{Timing Contraints and Schedulability} 
The programming language Real-Time Euclid has historical 
significance. Introduced as ``a language designed specifically to address reliability 
and guaranteed schedulability issues in real-time systems'', Real-Time Euclid 
focuses on providing a high level of exception and error handling coupled with 
forced time and space bounds \cite{real-time-euclid}. Consequently, Real-Time Euclid 
programs can be analysed rigorously to guarantee the schedulability of their 
processes. It therefore provides highly reliable software. 
However, the language has not been widely adopted.

\section{Summary}
This Literature Review has looked at three areas of relevance to this project.
First, the existing literature surrounding real-time systems has been examined,
with the different concepts defining a real-time system explored. Secondly,
abstractions that aid the development of real-time systems have been examined.
Finally, language design features of interest to real-time systems
have been documented. In the surrounding literature, there is a heavy focus on two
programming languages: Java and Ada. Although infrequent references also include C, 
there is little research seeking to expand the pool of languages in which 
real-time systems are developed. From this examination of existing programming 
languages, it is possible to determine requirements for real-time programming. 
The language must provide: concurrency, priority scheduling, access to a monotonic 
clock, an absolute sleep function, the Priority Ceiling and Priority Inheritance 
protocols, and finally a method of ATC. 
