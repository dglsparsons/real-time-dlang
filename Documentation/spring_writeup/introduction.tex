% Introduction to the report. 

% Word Count - 1000 words

% The scope of the project, setting the scene for the remainder of the report


% Scope of the project. 


% Setting the scene for the remainder of the project




% General introduction to the topic I'll be discussing  -- background
% information
% My 'Thesis statement' -- guess this means the aim goal? 
\section{General Introduction}
\subsection{What is a real-time system}
\subsection{Real-time systems programming languages}

% Why do i want to do it -- Importance.
% building upon previous work? Looking at something overlooked? Improving upon
% a previous research project? 
\section{Importance} % 200 words? 
\subsection{Existing Research}
\subsection{Expanding on Existing}

% Limitations -- be honest with the limitations of the project, i.e can only
% target certain operating systems, requires glibc, still has a garbage
% collector. 
\section{Limitations} % 200 words? 
\subsection{Operating System}
\subsection{External libraries}
\subsection{Garbage Collector}

% Assumptions. -- Assuming that the GC has negligible impact. ? 
\section{Assumptions} % 200 words? 
\subsection{Software Support}
\subsection{Hardware Support}
\subsection{Garbage Collector}


% Outline of what is to come in this report. 
\section{Report Outline} % 200 words
This report is divided into 5 main sections. 
First, a literature review will 
consider existing research, analysing what material already exists with regards 
to real-time systems, and situating this report within the context of these works. 
Secondly, a description and analysis of the problem is detailed, this sets out      %TODO- analysis?
an overview of interesting features of the D programming language, as well as 
describing features of real-time systems. This set of real-time features acts as 
the projects end goal. 
The report will then discuss the taken design approach to the implemented library,
discussing any decisions made during the libraries creation. It will explain the 
rationale behind implementation decisions. This section will also detail 
how sample programs may be written using the implemented library. 
The next section, Results and Evaluation, will consider two different properties 
of the provided library: a comparison with an existing real-time capable programming 
language, Ada, and a comparison between its own two alternative implementations 
of asynchronous transfer of control. 
Finally, conclusions will be drawn regarding the efficiency and safety of the 
two implemented asynchronous transfer of control methods, the effectiveness of 
real-time systems development in D over alternative language, and the future 
potential D has for developing real-time systems.

