% Introduction to the report. 

% Word Count - 1000 words

% The scope of the project, setting the scene for the remainder of the report


% Scope of the project. 


% Setting the scene for the remainder of the project

% General introduction to the topic I'll be discussing  -- background
% information
% My 'Thesis statement' -- guess this means the aim goal? 
%\section{General Introduction} % 200 words
This report sets out to detail and discuss the implementation of real-time 
programming capabilities and abstractions into the D programming language. 
Such support will take the form of a library, allowing a simplistic implementation 
into newly created programs, without the need to run custom compilers or make
modifications to the language's core libraries. Furthermore, the project has a design 
goal: any existing applications should not be affected through the introduction 
of this library. This means that any code written without prior knowledge should 
remain completely unaffected. 
\par\bigskip\noindent
The D programming language was chosen as the target for this project due to its 
inherent suitability for many real-time principals. The language shows a rich 
feature set, boasting high-level abstractions, low-level capabilities, inherent 
concurrency, improved memory safety over C/C++, a safe type 
system, native efficiency, built-in unit testing,  and a general purpose approach 
to programming. Simply stated, D aims to be a ``C++ done right'' \cite{qznc-tutorial}. 
With C++ seeing increasing use on large scale real-time systems, such as the F-35 
fighter jet, it follows a logical progression that D would provide a suitable 
alternative \cite{f35}.
\par\bigskip\noindent
Real-time systems are a classification of 
information processing system, in which input
stimuli require a response before the passage of a predefined deadline. Failure 
to meet this deadline may be a failure of the system. Input 
stimuli may be both externally generated i.e. by a trigger in the system's 
environment, or they may be released by the passage of time. Such systems are 
often safety critical, and failure to meet an imposed deadline may lead to 
an irrelevant result of no value to the system. It can also cause disastrous side-effects, 
such as the loss of life. Examples of such systems could include 
manufacturing processes, automotive vehicles, or even embedded sensors. 
% Why do i want to do it -- Importance.
% building upon previous work? Looking at something overlooked? Improving upon
% a previous research project? 
\section{Project Goal} % 200 words? 
This project aims to build upon previous research in the field of real-time systems. 
Significant study has gone into the schedulability of the task sets that 
define a real-time system, and also different programming languages capabilities 
to provide timing guarantees \cite{burns-sched-analysis,atc-article}. 
However, the evaluation of a new programming language's suitability for the 
development of real-time systems, and the provision of real-time programming 
abstractions to these languages is unexplored. This project aims to target that 
omission by extending any necessary programming abstractions to the D programming 
language, and detailing its suitability for future real-time systems programming.
% Limitations -- be honest with the limitations of the project, i.e can only
% target certain operating systems, requires glibc, still has a garbage
% collector. 
\section{Limitations} % 200 words? 
Throughout the course of this project, a few limitations have been placed on the 
use of D as a programming language for the development of real-time systems. 
This section aims to explain these restrictions, and the reasons for
their inclusion.
\subsection{Operating System}
Using D's existing features was a desired requirement for the project. D's native 
approach to concurrency, threads, do not provide a container in which they run. 
Instead, the language simply provides a wrapper around several different operating 
system calls, providing the correct type of thread for whichever operating system 
the program is provided for \cite{github-core-thread}. 
\par\bigskip\noindent
This has the advantage of flexibility with regards to operating system.
Furthermore, it does not require the use of a virtual machine, as seen in languages such as 
Java and C\# \cite{zhang2007exploiting}. However, this comes at a cost of implementation: 
different operating systems handle threads and scheduling very differently. 
In order to provide guaranteed latencies, real-time systems use a priority based 
scheduling approach. This approach is only available in D through the use of 
POSIX threads, targeting a POSIX compliant operating system such as GNU/Linux. 
\par\bigskip\noindent
As a result, the functionality provided by the real-time library for D is only 
applicable on POSIX compliant operating systems.

\subsection{External Libraries}

Similar to how D handles threads, many of the language's core features, such as 
mutexes and semaphores, are provided through interaction with external libraries. 
Many of the features implemented in the real-time library depend on the ability 
to interact with a standard C library, such as GlibC. This poses another limitation 
on the result of the project: the real-time features are only appropriate on 
operating systems with a standard C library provided. However, the use of C is 
very widespread, and C is supported on most operating systems. 

\subsection{Garbage Collector}
A further limitation is provided by D's use of a stop-the-world garbage collector. 
This, at any moment in time, may pause the execution of the entire program in 
order to perform memory management. As providing D with a real-time capable 
garbage collector is outside the scope of this project, an additional limitation 
is imposed. In order to guarantee the timely response of threads, and that no 
deadlines are missed, the garbage collector must either be avoided or memory 
management techniques such as pooling and preallocation must be used. 

% Assumptions. -- Assuming that the GC has negligible impact. ? 
\section{Assumptions} % 200 words? 
Due to the limitations described in the previous section, several assumptions were 
made during the course of this project. 
First, it is assumed that any software necessary needed to run a D program is 
provided. With regards to the development of real-time systems, this support  
extends to the assumption that a POSIX compliant operating system is provided, 
along with a suitable C library, such as GlibC. 
Additionally, it is assumed that the garbage collector will either be replaced, 
avoided, or that memory management techniques such as pooling and preallocation 
will be used so that garbage collections do not affect deadlines. 

\section{Statement of Ethics}
While it is unclear how the results of this project may be used, the project 
itself does not contain participants and as such there is no immediate ethical impact.

% Outline of what is to come in this report. 
\section{Report Outline} % 200 words
This report is divided into 5 main sections:
First, a literature review will 
consider existing research, analysing what material already exists regarding 
real-time systems, and situating this report within the context of these works. 
From this, a set of requirements for real-time programming will be generated. 
Secondly, a description and analysis of the problem is detailed: this sets out 
an overview of interesting features of the D programming language, as well as 
describing features of real-time systems. This set of real-time features acts as 
the project's end goal. 
The report will then discuss the taken design approach to the implemented library,
discussing any decisions made during the library's creation. It will explain the 
rationale behind implementation decisions. This section will also detail 
how sample programs may be written using the implemented library. 
The next section, Results and Evaluation, will consider two different properties 
of the provided library: a comparison with an existing real-time capable programming 
language, Ada, and a comparison between its own two alternative implementations 
of asynchronous transfer of control. 
Finally, conclusions will be drawn regarding the efficiency and safety of the 
two implemented asynchronous transfer of control methods, the effectiveness of 
real-time systems development in D over an alternative language, and the future 
potential D has for developing real-time systems.
