%Problem description and analysis for my project. 
% Word Count Guide - 1500 words.

This chapter aims to examine the literature review in order to provide a set of 
requirements for real-time programming. Following this, it will set out a 
description of the features of the D programming language, discussing its 
suitability for the development of real-time systems, and approach to crucial 
features.

\section{Requirements}
From the examination of existing languages in the literature review, it is possible 
to determine the criteria needed for real-time programming. These requirements are 
summarised in the following. Furthermore, additional desired features are set out 
in this section. 
\begin{table}[!htbp]
\centering
\begin{tabular}{|L{0.2\linewidth}|L{0.7\linewidth}|}
\hline
Requirement Number & 1 \\ \hline
Requirement & The ability to provide concurrent programming through a 
mechanism such as Java's Thread, or Ada's Task. \\ \hline
Rationale      & This is necessary in a real-time system to provide separation 
of concerns between separate components, and to  provide control over 
concurrent operations. \\ \hline
Criterion      & New threads of control can be created through a defined 
function. \\ \hline
Functional or  & Functional \\
Non-Functional & \\ \hline
Importance     & Necessary \\ \hline
\end{tabular}
\end{table}
\begin{table}[!htbp]
\centering
\begin{tabular}{|L{0.2\linewidth}|L{0.7\linewidth}|}
\hline
Requirement Number & 2 \\ \hline
Requirement & The ability to schedule concurrent threads based on their priority in 
                either a round-robin, or FIFO approach\\ \hline
Rationale      & This is necessary so that tasks with a shorter deadline, or more 
                important tasks are given priority, enabling them to meet their 
                deadlines. \\ \hline
Criterion      & A provided function must allow the scheduler to be changed \\ \hline
Functional or  & Functional \\
Non-Functional & \\ \hline
Importance     & Necessary \\ \hline
\end{tabular}
\end{table}
\begin{table}[!htbp]
\centering
\begin{tabular}{|L{0.2\linewidth}|L{0.7\linewidth}|}
\hline
Requirement Number & 3 \\ \hline
Requirement & Access a monotonic clock. \\ \hline
Rationale      & This is necessary in order to provide precise timing requirements 
                that are consistent, unlike a calendar clock. \\ \hline
Criterion      & A function must allow a time value to be read from a monotonic clock \\ \hline
Functional or  & Functional \\
Non-Functional & \\ \hline
Importance     & Necessary \\ \hline
\end{tabular}
\end{table}
\begin{table}[!htbp]
\centering
\begin{tabular}{|L{0.2\linewidth}|L{0.7\linewidth}|}
\hline
Requirement Number & 4 \\ \hline
Requirement & An absolute sleep statement \\ \hline
Rationale      & Relative sleep statements are subject to preemption, and may not 
                provide precise timing requirements. \\ \hline
Criterion      & A function must delay the calling thread until an absolute time. \\ \hline
Functional or  & Functional \\
Non-Functional & \\ \hline
Importance     & Necessary \\ \hline
\end{tabular}
\end{table}
\begin{table}[!htbp]
\centering
\begin{tabular}{|L{0.2\linewidth}|L{0.7\linewidth}|}
\hline
Requirement Number & 5 \\ \hline
Requirement & Implementation of the Immediate Priority Ceiling Protocol \\ \hline
Rationale      & In order to provide a limit to blocking a task may suffer, priority 
                inversion must be bounded. \\ \hline
Criterion      & The ability to specify a mutex as using the protocol\\ \hline
Functional or  & Functional \\
Non-Functional & \\ \hline
Importance     & Necessary \\ \hline
\end{tabular}
\end{table}
\begin{table}[!htbp]
\centering
\begin{tabular}{|L{0.2\linewidth}|L{0.7\linewidth}|}
\hline
Requirement Number & 6 \\ \hline
Requirement & Implementation of the Priority Inheritance Protocol \\ \hline
Rationale      & In order to provide a limit to blocking a task may suffer, priority 
                inversion must be bounded. \\ \hline
Criterion      & The ability to specify a mutex as using the protocol\\ \hline
Functional or  & Functional \\
Non-Functional & \\ \hline
Importance     & Necessary \\ \hline
\end{tabular}
\end{table}
\begin{table}[!htbp]
\centering
\begin{tabular}{|L{0.2\linewidth}|L{0.7\linewidth}|}
\hline
Requirement Number & 7 \\ \hline
Requirement & Implementation of Asynchronous Transfer of Control (ATC) \\ \hline
Rationale      & Real-time systems require ATC for a wide range of functionality, 
                including timing out computations, cancelling overrun deadlines,  
                or instantaneous mode changes of a system. \\ \hline
Criterion      & The ability to enter an abortable section of code, and remotely  
                abort the code. \\ \hline
Functional or Non-Functional & Functional \\ \hline
Importance     & Necessary \\ \hline
\end{tabular}
\end{table}
\begin{table}[!htbp]
\centering
\begin{tabular}{|L{0.2\linewidth}|L{0.7\linewidth}|}
\hline
Requirement Number & 8 \\ \hline
Requirement & No modifications to the languages compiler, or core libraries \\ \hline
Rationale      & In order to provide a more accessible system and to simplify the 
                process of developing real-time systems in D. Furthermore, this 
                keeps compatibility with existing D code, and allows any programs  
                written to keep up with the latest features and advances of D \\ \hline
Criterion      & For programming real-time systems in D, the default compiler and  
                an unmodified runtime may be used. \\ \hline
Functional or Non-Functional & Non-Functional \\ \hline
Importance     & Desired \\ \hline
\end{tabular}
\end{table}
\begin{table}[!htbp]
\centering
\begin{tabular}{|L{0.2\linewidth}|L{0.7\linewidth}|}
\hline
Requirement Number   & 9 \\ \hline
Requirement & Provision of tests for the provided library to show that 
                implementation is functioning correctly. \\ \hline
Rationale      & Provision of tests would allow an easier checking of the viability of  
                real-time features, and aids debugging in any potential future 
                modifications to the feature-set. \\ \hline
Criterion      & A set of unit-tests that ensure correct behaviour of the system. \\ \hline
Functional or Non-Functional & Functional \\ \hline
Importance     & Desired \\ \hline
\end{tabular}
\end{table}
\begin{table}[!htbp]
\centering
\begin{tabular}{|L{0.2\linewidth}|L{0.7\linewidth}|}
\hline
Requirement Number & 10 \\ \hline
Requirement & Provision of documentation, showing typical use cases for provided 
                functionality, and describing its usage. \\ \hline
Rationale      & In order to allow potential users to understand the purpose and 
                functionality provided, it must be explicitly shown in use, and
                documented.  \\ \hline
Criterion      & All functions and common interactions must be well documented \\ \hline
Functional or Non-Functional & Functional \\
Importance     & Desired \\ \hline
\end{tabular}
\end{table}
\FloatBarrier
\section{Analysis of Target Criteria} %100
This section aims to outline the functional requirements of the project,
offering a brief discussion on potential implementation for each of these requirements. 

\subsection{Concurrency and Priority Scheduling} %100
In order to provide scheduling of multiple different concerns through a
real-time system, the previous chapters discussed how the concurrent control of
system components is a crucial feature for real-time systems. D provides a
method of concurrency through its \texttt{Thread} class. 
\par\bigskip\noindent
Closely related to this concept, is the ability to schedule threads based on
their priority. In D, threads exist in a kernel space. Therefore, in order to
provide priority based scheduling, the systems scheduler requires changing. Burns
and Wellings show that there are several different scheduling algorithms that
may be used in real-time systems \cite{real-time-systems}. It is worth noting
that of all the operating systems that D runs on, only those that are POSIX
compliant are capable of handling such a means of priority scheduling. This
provides a limitation on the project: D can only be programmed in a real-time
capable manner on a POSIX compliant operating system.
By default, D is provided with no method to change the underlying operating
systems scheduler. 
The ability to adjust the scheduler to use either a Round-Robin, or FIFO approach 
is therefore a desired end-goal of the project.

\subsection{Monotonic Clocks} %100
As described in the Literature Review chapter, access to both a high precision
clock and absolute sleep method is necessary for a real-time system. D's
current implementation can access a high precision clock through the use of
functionality provided in the core runtime \cite{dlang-core-time}. However,
access to an absolute sleep function is missing. Instead, D's default sleep
method maps onto the C relative sleep function, \texttt{nanosleep}. There are
therefore two possible alternative implementations. Either the call to
\texttt{nanosleep} could be wrapped in such a manner that it is not possible to be preempted, 
or the C function, \texttt{clock\textunderscore{}nanosleep}, may be used. This function enables 
an absolute sleep with a user specifiable clock \cite{clock-nanosleep}. 

\subsection{Bounded Priority Inversion} %100
In order to achieve a bound on the amount of priority inversion that a task
might suffer two protocols exist: the priority inheritance protocol and the
priority ceiling protocol. These protocols have been detailed in the 
Literature Review section. D's implementation of mutexes currently has
neither of these protocols, and is therefore not suited for real-time systems.
However, similar to D's approach to the \texttt{Thread} class, the languages
core \texttt{Mutex} class provides a wrapper on POSIX mutexes
\cite{dlang-github-mutex}. The POSIX specification of mutexes states that
they can implement both the priority ceiling and the priority inheritance protocol 
\cite{mutex-setprotocol}. Therefore, there are two possible ways of
implementing the priority ceiling protocol and the priority inversion protocol
in D. Either the appropriate external C functions can be imported, allowing the
use of POSIX mutexes with a specified protocol, or the protocol itself may be
implemented on top of the existing \texttt{Mutex} implementation. 

\subsection{Asynchronous Transfer of Control} %100
Asynchronous Transfer of Control (ATC) is a difficult feature to implement due to
it's ``methodologically suspect'' nature \cite{Brosgol:2002:ATC}. As detailed
in the literature review Java, Ada and C with POSIX Threads all favour
different approaches to the implementation details. Simply stated, there are
three alternative implementations. In C, ATC can be simulated using the
\texttt{setjmp} and \texttt{longjmp} instructions. In Ada, a
two-threaded model is often used. This functions by blocking the calling thread
on a condition, while a subsidiary task is created to execute the abortable
code. In Java, an exception based approach uses the Thread \texttt{interrupt}
method to asynchronously throw an exception in the target thread. As D has the
ability to interact with C code, \texttt{setjmp} and \texttt{longjmp} may be
used. Furthermore, it's model of threads allows a thread cancellation based
approach to be taken. Finally, D has a similar model of exceptions to Java. By
combining D's exception model with POSIX signals, enabling the asynchronous
execution of code, D therefore has the potential to implement all three ATC
implementations. 




\section{Features of the D Programming Language}
As mentioned in the introduction, the D programming language has many 
characteristics that lend it nicely to the development of real-time systems.
These are outlined in further detail in this section. 
However, D also has contains certain characteristics that are less desirable for 
real-time systems, such as the use of a garbage collector. 
These less desirable features are similarly detailed in order to highlight the 
shortcomings of D. 

\subsection{High and Low-Level}
% Aims to achieve everything that is possible with other languages such as C++
% Contains high level abstractions
The D programming language is primarily inspired by C++, aiming to be a ``C++ done
right'' 
\cite{qznc-tutorial}. 
This explains the power that the language possesses. D is a high level, compiled 
language with support for object oriented programming. In addition, D boasts 
native efficiency: as with C and C++, there is no virtual 
machine hosting the languages environment and causing additional overhead. 
Furthermore, D aims to provide
powerful abstractions that aid in the speed and ease of development of general systems. 
The following code example, taken from dlang.org, cleanly and efficiently 
sorts lines of a file and outputs them. 
\begin{lstlisting}
import std.stdio, std.array, std.algorithm;

void main()
{
    stdin
        .byLineCopy
        .array
        .sort
        .each!writeln;
}
\end{lstlisting}
However, in addition to providing high-level abstractions, D maintains low 
level capabilities: the language still retains access to inline assembler, 
system calls, interrupts, and low level I/O.
\par\bigskip\noindent
This flexibility of D is one of its greatest advantages. It ``allows its user to tap 
into a wide range of programming techniques, without throwing too many 
idiosyncrasies in the way'' \cite{ddili-book}. 
These features also make D well suited to the development of real-time systems, 
as stated in the literature review, the ability to program sporadic tasks
may depend on the ability to handle interrupts in a timely and predictable manner.
Furthermore, the ability to perform IO operations using hardware allows the
system to interact with special purpose hardware, another crucial feature of
real-time systems.
D's multipurpose approach therefore allows the language to handle low-level concerns,
while simultaneously benefiting from code that can be aggressively optimized by the
compiler \cite{dlang-overview} and maintaining a readable and user-friendly
syntax. D is therefore a powerful and highly capable language.

\subsection{Concurrency}
%Message passing
D is often referred to as a language that implements the message passing model 
of concurrency, rather than the more standard lock and mutex approach 
\cite{ddili-book}. D's message passing API 
allows asynchronous messages to be transferred between threads of control using 
message boxes. This allows a high-level of abstraction and provides an 
interface that is simplistic to use. However, it is also misleading: the underlying 
model of D uses the more typical lock and mutex approach 
\cite{dlang-github-mutex}, 
with the message passing interface simply built on top of this 
\cite{dlang-github-concurrency}. 
As such, when developing in D, the programmer has the option to choose between 
the higher level message passing interface or the lower level lock and mutex 
approach. It is even true that a mix approach can be taken, using both.
\par\bigskip\noindent
D also has an interesting approach to sharing memory between threads. In D 
global variables are not shared between different threads as they are in 
other languages. Instead, each thread has a local copy of the global variable. 
This is a concept that is known as Thread Local Storage (TLS) 
\cite{migrate-to-shared}. 
If resources are to be shared between different threads, D provides a way of 
achieving this: the \texttt{shared} and
\texttt{\textunderscore{}\textunderscore{}gshared} keywords.
\par\bigskip\noindent
The core runtime for D provides an additional interesting feature: an atomicop 
library 
\cite{core-atomic}. 
This provides a mechanism for some degree of sharing to be performed without the 
need for locks, allowing many operations to be performed indivisibly.

\subsection{Memory Safe Programming}
% Concept of memory safe programming that D supports.
An area of particular interest to real-time systems is the notion of memory 
safe programming. In addition to providing powerful inbuilt types such as arrays 
in D, the D compiler also implements range checking and numerous other 
safety checks on compilation, preventing any undefined behaviour from occurring.
Additional safety features are also provided in the SafeD subset of the language
\cite{safe-d}. The SafeD subset will only allow functions to perform 
operations that are deemed as either safe or trusted, preventing any 
potential corruption of memory. Examples of unsafe functions could include the 
manual manipulation of pointers, or the use of unsafe types. A major difference 
with regards to other languages is that D provides sufficient power so that many complex 
operations can be performed and used, while still remaining completely safe.

\subsection{Garbage Collector}
%GC
Coupled with the safety of the SafeD subset, D provides additional memory
safety through its use of a garbage collector. A garbage collector
automatically manages heap allocated memory. It is responsible for managing
the lifetime of data, and cleaning up unused memory. 
However, this safety comes at a cost. One significant problem with developing a 
real-time system in D is the need to bypass D's garbage collector. 
The current implementation of the garbage 
collector is a `stop the world' approach: when garbage collection is occurring, 
all threads are halted from making progress 
\cite{dlang-garbage}. This causes problems with 
real-time systems. A real-time system must be able to guarantee a maximum response 
time on any operation performed, predictable behaviour is therefore fundamental to this. 
With a stop the world garbage collector, such behaviour cannot be guaranteed. 
\par\bigskip\noindent
One possible solution to this problem is to disable the garbage collector for 
real time applications. This is possible to do by marking functions \texttt{@nogc}, 
or by calling \texttt{GC.disable()}. However, this is not without drawbacks: 
many of D's higher level features and libraries use garbage collection. 
Another possibility is to replace D's garbage collector with a real-time 
garbage collector. This would provide predictable behaviour, and therefore
would enable precise timing requirements to be met.
However, due to the amount of further work required, tackling the issue of the garbage 
collector is outside the scope of this article.
A third possibility of avoiding the garbage collector comes in through
detaching threads from D's runtime. This prevents the garbage collector from
stopping them. However, any garbage collected objects used must be registered
by another thread within the runtime to prevent them being collected. 
The final possibily is to preallocate all memory that could be needed. Using
pooling this would prevent applications from requesting more memory during
their execution, and thus prevents garbage collections. 

\subsection{Additional Interesting Features}
% Idea of built in tests, contract programming, error handling
Another unique feature of D is the ability to build unit tests into a 
program without the need for external libraries: D provides support for 
unittesting, complete with coverage analysis, as a language feature. 
\par\bigskip\noindent
In addition, D provides a Java style model for exceptions and errors, enabling 
a much simpler error handling system than the C style ERRNOs 
\cite{ddili-book}. 
These exceptions are functionally very similar to Java exceptions. Compared to
C++'s exception model, D throws exceptions through reference, instead of by
value \cite{interface-to-cpp}. 
